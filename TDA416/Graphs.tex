%General
\documentclass{article}
\usepackage[utf8]{inputenc}

%Symbols
\usepackage{commath}
\usepackage{amsmath}

%Formatting
\usepackage{hyperref}
\newtheorem{treedef}{Tree definition}
\hypersetup{colorlinks=true}
\usepackage{graphicx}
\graphicspath{ {img/} }
\usepackage{caption}

\title{Graphs}
\date{2016-02-23}
\author{Hjort}

\begin{document}
  \pagenumbering{arabic}
  \maketitle

  \section{Asymptotic properties}

      \subsection{For the graph itself}

      $$\abs{V} = n \text{ (order)}$$

      $$\abs{E} = e \text{ (size)}$$

      We assume these are not \href{https://en.wikipedia.org/wiki/Multigraph}{multigraphs}.

      $$ \abs{E} \in O(\abs{V}^2) $$

      $$\abs{E} \leq \cfrac{n(n-1)}{2}$$

      $$\abs{E} \leq n^2 - n \text{ (for undirected graphs)}$$

      \subsection{For the represenation}

      Adjacency matrix (efterföljarmatris) has size 
      
      $$ \in O(n^2)$$

      Adjacency list (efterföljarlista) has size
      
      $$\in O(n + e)$$

      \section{Tree (fritt träd)}

      \begin{treedef}
          Undirected, acyclic and connected graph
      \end{treedef}

      \begin{treedef}
          A spanning tree for a graph $G = (V, E)$ is a free tree
          $T=(V,E^{\prime})$ where $E^{\prime} \subseteq E$ 
      \end{treedef}

      Thus, to create a spanning tree from a connected graph, we remove all
      cycles.
      
\bibliography{Bibl} 

\bibliographystyle{ieeetr}

\end{document}
