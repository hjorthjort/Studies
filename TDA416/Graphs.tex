%General
\documentclass{article}
\usepackage[utf8]{inputenc}

%Symbols
\usepackage{commath}
\usepackage{amsmath}

%Formatting
\usepackage{hyperref}
\usepackage{amsthm}
\newtheorem{definition}{Definition}
\newtheorem{theorem}{Theorem}
\hypersetup{colorlinks=true}
\usepackage{graphicx}
\graphicspath{ {img/} }
\usepackage{caption}

\title{Graphs}
\date{2016-02-23}
\author{Hjort}

\begin{document}
\pagenumbering{arabic}
\maketitle

\section{Asymptotic properties}

  \subsection{For the graph itself}

      $$\abs{V} = n \text{ (order)}$$

      $$\abs{E} = e \text{ (size)}$$

      We assume these are not \href{https://en.wikipedia.org/wiki/Multigraph}{multigraphs}.

      $$ \abs{E} \in O(\abs{V}^2) $$

      $$\abs{E} \leq \cfrac{n(n-1)}{2}$$

      $$\abs{E} \leq n^2 - n \text{ (for undirected graphs)}$$

  \subsection{For the represenation}

      Adjacency matrix (efterföljarmatris) has size 
      
      $$ \in O(n^2)$$

      Adjacency list (efterföljarlista) has size
      
      $$\in O(n + e)$$

\section{Tree (fritt träd)}

  \begin{definition}
      Undirected, acyclic and connected graph
  \end{definition}

  \begin{definition}
      A spanning tree for a graph $G = (V, E)$ is a free tree
      $T=(V,E^{\prime})$ where $E^{\prime} \subseteq E$ 
  \end{definition}

  Thus, to create a spanning tree from a connected graph, we remove all
  cycles.

\section{Searching}

    \subsection{Depth First Search (DFS)}

        Equivalent to pre/post/inorder search in the case of trees.

        Since there are cycles, we could potentially loop through nodes
        forever. Thus, we have to flag visisted nodes somehow.

        \begin{enumerate}
            \item for every node $v \in V$
            \item \hspace{10pt} $v.visited = false$
            \item as long as there are unvisited nodes
            \item \hspace{10pt} pick one unvisited, $v$
            \item \hspace{10pt} $dfs(v)$
        \end{enumerate}

        \subsubsection(Time complexity)
        Time complexity for DFS is $O(n+e)$ since we visit every node once, and 
        check every edge from every node to see if they lead to a previously 
        unvisited node.

        Expressed in only one variable.

        $$\text{(only n) } e < n^2 \implies O(n+e) \subseteq O(n + n^2) =
        O(n^2)$$

        $$\text{(only e, for connected graph) } n < e + 2 \implies O(n + e)
        \subseteq O(e + e) = O(e)$$

    \subsection{Breadth First Search (BFS)}

        
\section{Minimal Spanning Trees (MST)}

    \begin{definition}
        An MST for a weighted graph is a spanning tree for which the sum of the
        weights of the edges is the smallest possible.
    \end{definition}

    \begin{theorem}
        Consider 2 disjoint set of vertices in a graph, $S$ and $V \setminus
        S$, such that they have a spanning tree that is also part of the MST
        for the entire graph. Now every edge between these two sets that has
        minimal weight – that is, minimal among these specific edges – will be
        part of a minimal spanning tree.
    \end{theorem}

    \begin{proof}
        If we consider each of the sets of vertices and the edges between them
        as separate graphs, then we can create two spanning trees for the sets
        $T_S$ and $T_{V \setminus S}$. An MST for the whole graph must consist
        of some version of these two trees, and exactly one edge connecting
        them, which must be a minimal edge.
    \end{proof}

    This will be used when constructiong Prim's algorithm.

\bibliography{Bibl} 

\bibliographystyle{ieeetr}

\end{document}
