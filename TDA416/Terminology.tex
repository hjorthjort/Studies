%General
\documentclass{article}
\usepackage[utf8]{inputenc}

%Symbols
\usepackage{commath}
\usepackage{amsmath}

%Formatting
\usepackage{hyperref}
\usepackage{amsthm}
\newtheorem{definition}{Definition}
\newtheorem{theorem}{Theorem}
\hypersetup{colorlinks=true}
\usepackage{graphicx}
\graphicspath{ {img/} }
\usepackage{caption}

\title{Datastructures and algorithms - Terminology}
\date{2016-03-11}
\author{Hjort}

\begin{document}
\pagenumbering{arabic}
\maketitle

\section{Graphs}

\textbf{Sources: \href{http://www.cse.chalmers.se/edu/course/tda416/mtrl/lectures/F9.pdf}{Slides} and \href{https://en.wikipedia.org/wiki/Glossary_of_graph_theory}{Wikipedia}.}

\begin{tabular}{l || l || p{5cm}}

    English & Swedish & Definition \\ \hline\hline
%--------------------------------------------------
    component & komponent & a connected "sub-graph" of any graph. A connected graph has one component. \\\hline
    degree & grad & the number of edgeds connecting to a vertix \\\hline
    in-degree & ingrad & the number of edgeds that end in a vertix (directed graphs only)\\\hline
    out-degree & utgrad & the number of edgeds that begin in a vertix (directed graphs only)\\\hline
    density & densitet & the number of edges divided by the number of potential edges \\\hline
    density (undirected) & - & $\dfrac{ 2 * \lvert E\rvert }{ \lvert V\rvert * (\lvert V\rvert - 1)}$ \\\hline
    density (undirected) & - & $\dfrac{ \lvert E\rvert }{ \lvert V\rvert ^2}$ \\\hline
    edge & båge & - \\\hline
    order & ordning & the number of vertices in the graph \\\hline
    size & storlek & the number of edges in the graph \\\hline
    vertix & nod & - \\\hline
    
\end{tabular}

\begin{enumerate}
    \item A
    \item numbered
    \item list
\end{enumerate}

\subsection{Symbols}

\subsubsection{Sets}

\begin{align*}
    & \subset \text{Proper subset} \\
    & \subseteq \text{Subset} \\
    & \setminus \text{\textbackslash} \\
\end{align*}

\bibliography{Bibl} 

\bibliographystyle{ieeetr}

\end{document}
