\documentclass{article}
\usepackage{fullpage}
\title{Reding \emph{The Gender Agenda in Computer Ethics} by Alison Adam}
\author{Rikard Hjort}
\begin{document}
\maketitle

% Quick summary
Adam's chapter acts as an introduction to feminist theory for the field of
computer ethics. As such, it introduces some standard feminist critique of
technical fields and unquestioned assumptions of gender that often hide in and
influence the research results.

The chapter argues that the field of feminist theory, and feminist ethics in
particular, has much to offer the field of computer ethics. Adam argues for are
the critical analysis of some of the quantative studies on ethics and decision
making, and their gender dimensions, that has been made in business and
management literature. In particular, she is critical of the unquestioned
positivism that allows authors to hide possible biases, e.g., relating to
gender, behind a veil of objectivity by using quantative methods, while not
questioning how the study is conducted, and how power relations could reasonably
influence the results. According to Adam, the debate over quantative vs.
qualitative has been raging within the social sciences for decades, and in the process
much nuance has developed, and much deep reasoning which can be applied when evaluating
the results of the most prominent studies on male and female ethics. However,
this reasoning and justification of methodology is missing in the scrutinized
studies, and as a result, their conclusions come under question. The main issue,
according to Adam, is an ``undertheorizing of gender and ethics''\footnote{pp.
  601}, partly caused by the fragmentation of the research space -- the studies
in question seem unaware of each other, as well as of much of feminist research.
This leads to some less informed conclusions being drawn, seemingly based on
gender stereotypes, which may not be at all warranted by the statistical findings.

Adam goes on to give some remarks on how a feminist view of gender and computing
can help with reform to attrat women to computing, and argues that it would be
disrable, especially as the exclusion of women, wether intended or not, means
they are not present for conversations that impact society to high degree.

The penultimate part of the chapter is an application of feminist theory to concrete
issues in computer ethics: cyberstalking and hacking. It is a challenge of the
egalitarian ethos of the hacker community (hacking taken to mean both the act of
making software, and the more illicit activities of breaking into computer
systems). The main piece of work Adam scrutinizes is a book on hacker
communities, and Adam gives a fairly standard feminist critique of the work --
women are presented first and foremost in their relation to men, as wives, and
that stating the ethos that everyone is welcome and judged by their skill in a
community does not automatically make it egalitarian and meritocratic.

Adam concludes by viewing the field of ``feminist computer ethics'' from the
reverse angle: how computer ethics infused with feminist critique can enrich
feminist ethics. Thanks to Donna Haraway's seminal paper \textit{A Cyborg
  Manifesto}, there is already a large body of ideas in the intersection of
thoeries of gender, feminism and technology, especially around technology as
an enabler of a post-gender world. A popular term for the feminism at this
intersection is \textit{cyberfeminism}, and is largely centered around studying
and theorizing about female hackers as examples of women controlling technology,
and thereby changing power dynamics.

\paragraph{}The reason that I chose this chapter is that I have a background in
Gender Studies, and skimming the chapter it seemed to have an interesting
approach. I was a little bit disappointed, though, in the rather broad but
shallow treatment of the items of critique -- the text seemed to be intended foremost as an
introductory text to feminist critique for scholars of computer ethics. I think
it was a clear, albeit somewhat meandering introduction, and it di a good job of
articulating important ideas clearly. However, I am in the unfortunate position
as a reader of being familiar with the methods being introduced, and perhaps it
did not give me as much value for my time as some other chapter might have.

My only real criticism of the arguments in the chapter is that is showed a
common bias that I see in a lot of gender discussion on
on STEM subjects, viz. that it is assumed that the lack of women in STEM (in
this case computing) is their loss: ``women are still absent from employment in
well-paid and interesting careers'', a framing which hides the fact that other
well-paid and interesting careers with more human interaction in fields such as law,
governance and medicine, where women are not underrepresented. There may be
something unattractive about STEM subjects for people of high social competence,
or something attractive about it for people of lower social competence, a trait
that seems to correlate with gender.

I will take with me Adam's description\footnote{pp. 591-592} of Carol Gilligan's book \textit{In a
  Different Voice}, and her result that men and women, for whatever reason, tend
to use different means to arrive at ethical decision, women to a lesser degree
leaning on law and Kantian moral philosophy, while men rarely consider
relational aspects when making such decisions. Apparently, Gilligan spent spent
some time refuting the work of Lawrence Kohlberg which was published the year
before, and his outline of moral development, with Kantian, rule-based ethics
being the highest form of ethics, which Gilligan argues is actually very much a
gendered hierarchy, which mostly justifies traditional masculine ethics rather
than a universal value system.
\end{document}


TASK
http://www.cems.uwe.ac.uk/~pchatter/2011/pepi/The_Handbook_of_Information_and_Computer_Ethics.pdf

Choose a chapter that interests you, read it, and write a short report:

    Summarize the contents of the chapter
    Why do you find this chapter interesting?
    Will this chapter be helpful in making ethical choices in real situations?

I’m expecting a report of about one A4 page.

Chapters shortlist:
* Online Anonymity
* Responsibilities for Information on the Internet
* The Ethics of Cyberconflict
* Regulation and Governance of the Internet [24 pages]
* Information Overload [23 pages]
* Cencorship and Access to Expression [17 pages]
* The Gender Agenda in Computer Ethics [33 pages]
* The Digital Divide: A Perspective from the Future [19 pages]