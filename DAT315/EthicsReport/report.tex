\documentclass{article} \usepackage{fullpage} \title{Reding \emph{The Gender
    Agenda in Computer Ethics} by Alison Adam} \author{Rikard Hjort}
\begin{document}
\maketitle

% Quick summary
Adam's chapter acts as an introduction to feminist theory for the field of computer ethics and argues that feminist theory, and feminist ethics, in particular, can advance computer ethics. Adam makes a critical analysis of some influential quantitative studies on ethics, decision making, and gender published in business and management literature. In particular, she is critical of the unquestioned positivism that allows authors to hide possible biases, e.g., relating to gender, behind a veil of objectivity. A regular quantitative study of students is a common method, and Adam argues that power relations could reasonably influence the results in ways the authors do not reflect on.\footnote{p. 596}

The debate over quantitative vs. qualitative methods has been raging among social scientists for decades and the pros and con have been thoroughly scrutinized. However, this reasoning and justification of methodology are mostly missing in the present studies\footnote{p. 597}, and as a result, their conclusions come under question. The main issue, according to Adam, is an ``undertheorizing of gender and ethics''\footnote{pp. 601}, partly caused by the fragmentation of the research space -- the studies in question seem unaware of each other, as well as of much of feminist research. This leads to some less informed conclusions being drawn, seemingly based on gender stereotypes, which may not be at all warranted by the statistical findings, while still seeming obvious to the authors, and apparently also their reviewers.

Adam goes on to give some remarks on how a feminist view of gender and computing can help to reform computing to attract more women and argues that this is a valid goal, especially as the exclusion of women, whether intended or not, means they are not present for conversations that impact society to a high degree.\footnote{This is the only part of the chapter I want to criticize: Adam shows what I perceive to be a common bias, often present in discussions on gender in the STEM subjects, viz. that it is assumed that the lack of women in STEM (in this case computing) is their loss. Adam writes that ``women are still absent from employment in well-paid and interesting careers'' [p. 605], a framing which hides the fact that other well-paid and interesting careers with more human interaction in fields such as law, governance, and medicine, where women are not underrepresented in the population entering these fields. It is far from sure, but still important to consider, that that there may be skills that are more prevalent among women that offer them more opportunities when deciding on a career than their male peers, and that STEM simply does not rank that high for people with many options.}

The sixth section of the chapter\footnote{pp. 606-611} is an application of feminist theory to concrete issues in computer ethics: cyberstalking and hacking. It is a challenge of the egalitarian ethos of the hacker community (hacking taken to mean both the act of making software and the illicit activity of breaking into computer systems). The main piece of work Adam scrutinizes is a book on hacker communities, and Adam gives a fairly standard feminist critique of the work, on two points: that presenting women first and foremost in their relation to men, as wives, dehumanizes them and obscures their capabilities; and that stating the ethos that everyone is welcome and judged by his or her skill in a community does not automatically make it egalitarian and meritocratic.

Adam concludes by viewing the field of ``feminist computer ethics'' from the reverse angle: how computer ethics infused with feminist critique can enrich feminist ethics.\footnote{pp. 611-615} Thanks to Donna Haraway's seminal paper \textit{A Cyborg Manifesto}, there is already a large body of ideas in the intersection of theories of gender, feminism, and technology, especially around technology as an enabler of a post-gender world. A popular term for feminism at this intersection is \textit{cyberfeminism}, and there are several papers considering female hackers as examples of women controlling technology, and thereby changing power dynamics.

\paragraph{}The reason that I chose this chapter is that I have a background in Gender Studies, and skimming the chapter it seemed to have an interesting approach. I was somewhat disappointed in the rather broad and basic treatment of the items of critique -- the text seemed to be intended foremost as an introductory text of feminist critique and computer ethics to scholars of both fields. I think it was clear and did a good job of articulating important ideas clearly. However, I am in the unfortunate position as a reader of being familiar with the methods being introduced, and perhaps it did not give me as much value for my time as some other chapter might have.

My main takeaway is Adam's description\footnote{pp. 591-592} of Carol Gilligan's book \textit{In a Different Voice}, and her result that men and women, for whatever reason, tend to use different means to arrive at ethical decision, women to a lesser degree leaning on law and Kantian moral philosophy, while men rarely consider relational aspects when making such decisions. Apparently, Gilligan spent some time refuting the work of Lawrence Kohlberg which was published the year before, and his outline of moral development, with Kantian, rule-based ethics being the highest form of ethics, which Gilligan argues is actually very much a gendered hierarchy, which mostly justifies traditional masculine ethics rather than a universal value system. I have myself been guilty of falling into the trap of reasoning like Kohlberg, and it will be very healthy to remember this critique when faced will ethical dilemmas, in whatever form, in my practice as a computer scientist in (hopefully) diverse professional contexts, and in my life in general. I will make sure to not dismiss relational arguments but take care to reassess my own stances, which tend to be more rule-oriented.
\end{document}


TASK
http://www.cems.uwe.ac.uk/~pchatter/2011/pepi/The_Handbook_of_Information_and_Computer_Ethics.pdf

Choose a chapter that interests you, read it, and write a short report:

    Summarize the contents of the chapter Why do you find this chapter
interesting? Will this chapter be helpful in making ethical choices in real
situations?

I’m expecting a report of about one A4 page.

Chapters shortlist: * Online Anonymity * Responsibilities for Information on the
Internet * The Ethics of Cyberconflict * Regulation and Governance of the
Internet [24 pages] * Information Overload [23 pages] * Cencorship and Access to
Expression [17 pages] * The Gender Agenda in Computer Ethics [33 pages] * The
Digital Divide: A Perspective from the Future [19 pages]