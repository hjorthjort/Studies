%General
\documentclass{article}
\usepackage[utf8]{inputenc}
\usepackage{fullpage}
\usepackage[swedish]{babel}
\selectlanguage{swedish}

% Supress warning of hboxes in bibliography.
\usepackage{etoolbox}
\apptocmd{\sloppy}{\hbadness 2057\relax}{}{}
\apptocmd{\sloppy}{\hbadness 10000\relax}{}{}

% Bibliography
\usepackage{apacite}
\usepackage{multibib}
\newcites{one}{Referenser}
\newcites{two}{Referenser}
\newcites{three}{Referenser}
\newcites{four}{Referenser}

%Symbols
\usepackage{commath}
\usepackage{amsmath}
\usepackage{amssymb}

%Numbering
\usepackage{chngcntr}
\counterwithin{figure}{section}

%Formatting
\usepackage{bussproofs}
\usepackage{amsthm}
\usepackage{mathtools}
\usepackage{graphicx}
\graphicspath{ {img/} }
\usepackage{caption}
\usepackage{hyperref}
\hypersetup{breaklinks=true}

\title{ITS023 Hemtentamen 2016}
\date{\today}
\author{Rikard Hjort} 
\begin{document}
\maketitle
% Tempot och stilen i denna är bra: https://pingpong.chalmers.se/courseId/6794/node.do?id=3289921&ts=1475749801574&u=1825887321

\section{Bekämpa denguefeber söder om Sahara}

\subsection{Problemläget}

Det är inte helt klart hur vanligt denguefeber är i Afrika \citeone[s. 8]{WHO:2009}. Redan detta är ett stort hinder för arbetet med bekämpning, eftersom man inte vet var insatser behövs mest, vilket gör att resurser inte används så effektivt som möjligt; utbildning och stora dräneringsprojekt kan vara kostsamma insatser. Denguefeber framstår inte heller som ett lika allvarligt problem som som vanligare och dödligare sjukdomar som HIV/AIDS \citeone[s. 8]{WHO:2009}. Detta gör det svårare att få stöd till insatser och att övertyga människor i riskområden att anpassa sina vanor för att minska spridningen. De åtgärder som krävs på individnivå för att förstöra habitat för mygggor, främst \emph{Aedis aegypti} \citeone[s. 60-161]{WHO:2009} kommer vara svåra att få till om inte gemene man ser problemet som prioriterat.

Det är också svårt att övertyga människor om vikten av att bekämpa denguefeber om man inte vet hur vanligt det är, var det håller på att spridas, etc. Därför behöver informationsbristen lösas för att kuna bedriva effektivt påverkansarbete.

\subsection{Åtgärder}

Jag föreslår därför två IT-plattformar med tillhörande databaser: en för att samla och en för att sprida information. Plattformarna görs tillgängliga via API:er, och betrodda aktörer kan uppdatera datan i plattformarna. På så sätt kan man göra plattformarna tillgängliga på det sätt som passar det lokala kontextet: SMS-tjänst, smarthpone-app, webbapplikation eller datorprogram.

\subsubsection{Samla information}

Den första plattformen (Plattform 1) har databaser för betrodda aktörer: en för att registrera bekräftade och misstänkta sjukdomsfall; en annan registrerar vaccinationer; ytterligare en innehåller platsdata om områden som haft förhöjd risk, platser där stillastående vattensamlingar bildas, och information om städer, till exempel hur sophanteringen fungerar. 

Målet med Plattform1 är att hjälpa WHO, myndigheter och hälsovårds\-apparaten att fatta informerade beslut om var insatser behövs – särskilt vid plötsliga utbrott, då det är försent att börja samla information, men den är som mest nödvändig. Plattform 1 skulle också kunna användas till forskning om denguefebers spridning, vilket skulle underlätta hjälparbetet ytterligare. Slutligen skulle Plattform 1 också kunna användas för att informera allmänheten, vilket vi återkommer till.

\subsubsection{Sprida information}

Den andra plattformens (Plattform 2) data görs öppet tillgänglig. Kärnan i den är en databas med de bästa råden för hur man förstör potentiella habitat för \emph{Aedis aegypti}. Till exempel föreslår WHO en rad specifika insatser \citeone[s. 60-64]{WHO:2009}. Plattformen kan utnyttja aktuell väderdata och rapporter om dengueutbrott, till exempel för automatiska utskick eller push-notifikationer till berörda parter (hushåll, arbetsplatser, myndigheter, etc.) när riskerna ökar.

Ovanpå plattformen byggs sedan applikationer till smarttelefoner, hemsidor, och SMS-tjänster som lokala regeringar och hjälporganisationer kan använda för att skicka ut information när den behövs, dit den behövs. Efter kraftiga regnfall kan man till exempel uppmana befolkningen att åtgärda vattensamlingar, till exempel med SMS, eller med videoklipp som i Singapore \citeone{Singapore:video}.
 
Plattform 2 kan också läsa data från Plattform 1 för att upptäcka ökade risker. Ett bra exempel är Singapores karta över ''aktiva kluster'' \citeone{Singapore:clusters} och övervakningsdata från Puerto Rico \citeone[s. 73]{WHO:2009}. Plattform 1 kan på så sätt ha en viktig påverkan på medvetenheten om denguefeber, vilket är högprioriterat, vilket gör att jag betraktar en Plattform 1 som den högst prioriterad.


\subsubsection{Malaria}
% Lokalt kontext %
Malaria är en sjukdom som ofta överskuggar denguefeber \citeone[s. 8]{WHO:2009}. I samhällen där malaria är vanligt bör kampanjer ta hänsyn till detta. Det kan vara väldigt olämpligt att fokusera på denguefeberprevention och bättre att fokusera på malariaprevention, eftersom även malaria orsakas av myggor. Genom att anpassa sig till det lokala kontextet och i första hand fokusera på malariaprevention kan informationen – hur man förstör habitat för myggor – få maximalt genomslag.

\subsubsection{Samarbete med lokala myndigheter}
% Intressenter %
Vår grupp har som mål att utrota denguefeber. Det går dock inte att bortse från att lokala myndigheter kan prioritera andra sjukdomar högre, och inte vill lägga resurser på att använda våra lösningar. För att nå våra mål bör vi därför erbjuda oss att utbilda medicinsk personal i datainsamling. Vi behöver också göra det mycket enkelt för lokala myndigheter att bidra, genom att bygga integrationer med deras nuvarande sjukvårdssystem. Det kan till exempel röra sig om att vi extraherar data från inscannade papper från institutioner som främst använder papper.

% Inflytandesfär %
Som överstatligt organ är vi begränsade i vad vi kan göra, och vår roll bör vara att skapa förutsättningar för och bedriva påverkansarbete mot de lokala aktörerna. Vi har antagligen inte resurser att på egen hand undersöka och ta prover för att avgöra förekomsten av denguefeber. Det är något vi måste göra i samarbete med lokala myndigheter, genom att få utbilda medicinsk personal i lämplig provtagning och datainsamling, och i hur man kan använda de IT-verktyg vi tar fram.

% Bristande information
Vi vet inte än hur lokala aktörer skulle ställa sig till plattformarna, men det är de som förväntas använda dem. Jag föreslår att vi sätter ihop arbetsgrupper med representanter för WHO, lokala myndigheter, medicinsk personal och andra intressenter, för att få återkoppling på hur plattformarna bör se ut, och om de ens ska byggas. Innan dess bör vi inte påbörja någon utveckling, för att inte riskera att bygga fel saker.

\subsection{Följdproblem av åtgärder}

% Följdproblem %
Eftersom vi arbetar med en mycket stor region, i vilken det finns länder med många olika regimtyper och statsapparater, finns risken att sådana plattformar används för övervakning och desinformation. Om en stat till exempel använder vår plattform för att få reda på sina medborgares sjukhusbesök, eller sprida falsk data i skrämselsyfte, kan förtroendet för båda WHO och för statlig hälsoinformation kraftigt undermineras, vilket gör datainsamling och informationsspridning mycket svårt.

\subsection{Osäkerhetsparametrar}

Den höga korruptionsgraden i till exempel Angola \citeone{corruption:angola} gör att det kan vara svårt och riskabelt att få med stater i vårt datainsamlingsarbete, eftersom vi inte vet om de ser Plattform 1 som en lämplig väg att sprida desinformation, eller om de alls vill delta. Jag föreslår därför att vi börjar använda Plattform 1 i ett utvalda stater och gradvis expanderar till fler, efter utvärdering.


\bibliographyone{hemtenta} 

\bibliographystyleone{apacite}


\clearpage
\section{Livscykelanalys}
% 1118 ord!

\subsection{Fördelar}

Nästan all mänsklig nytta kan genereras på mer än ett sätt. En anteckning kan till exempel göras med en enkel blyertspenna eller reservoirpenna, men en resrvoirpenna håller längre. De olika alternativen har inte bara olika stor miljöpåverkan, utan också olika sorters miljöpåverkan: där en reservoirpenna kräver smådelar av plast och metall, bläck och mycket energi, kräver en enkel blyertspenna trä och grafit. 

Den stora behållningen av LCA är möjligheten att mäta båda olika sorters miljöpåverkan och storleken på miljöpåverkan i förhållande till nyttan som varan eller tjänsten skapar, mätt i ''funktionell enhet''. Den kan dessutom avgöra vad som har en stor miljöpåverkan respektive liten. LCA tar dessutom hänsyn till hela kedjan av nyttoproduktion, ''från vaggan till graven'', så att man i siffor kan se den totala miljöpåverkan man orsakar om man väljer ett viss alternativ för nyttoproduktion istället för ett annat, och även identifiera var det finns störst potential till förbättring \citetwo[s. 95]{la} \citetwo[s. 21]{hitch}.

En annan fördel med LCA är att det finns standarder och konventioner kring hur LCA bör utföras, vilket skiljer det från verktyg som är interna i ett företag eller organisation på så sätt att fler människor kan förstå och granska en given LCA, vilket åtminstone i teorin innebär att det ställs höga krav på objektivitet i en LCA. Den allmänt accepterade strukturen för en LCA tillsammans med en gemensam terminologi underlättar diskussionen kring LCA \citetwo[s. 95]{la}. Det gör också att LCA kan användas vid inköp där köparen anser att miljöpåverkan är en viktig aspekt \citetwo[s. 176]{dahlin}.

\subsection{Nackdelar}

Lindfors och Antonsson nämner en rad nackdelar med LCA. Till exempel säger de att LCA ger en ''översikts\-bild, som alltid innehåller osäkerheter'', att de många olika metoder som går under namnet LCA gör att olika LCA:er kan ge olika resultat, att avsaknad av eller lågkvalitativ data ger stora osäkerheter och att även om LCA-metoden syftar till att vara så helttäckande som möjligt, så går det aldrig att täcka in alla aspekter \citetwo[s. 101]{la}.

Vidare nämnde Marcus Wendin i sin föreläsning att LCA inte tar upp lokala aspekter, såsom hur stora risker eller hur mycket miljöförstöring ett område utsätts för, utan alla skadeverkningar buntas ihop i totaler. Han nämnde också att en LCA förstås är väldigt tidskrävande eftersom den innehåller många steg, där vart och ett måste utföras noggrant. Till exempel behöver man i inventeringsanalysen ta hänsyn till \emph{alla} material- och energiflöden in i systemet, och likaså alla utflöden.

En LCA säger inte heller någonting om vad som \emph{borde} göras, utan kan bara beskriva hur stor miljö\-påverkan är eller skulle kunna vara. Inte heller tar en LCA hänsyn till aspekterna ekonomisk eller social hållbarhet. Till detta kan man istället använda livscykelkostnadsanalys respektive social livscykelanalys \citetwo[s. 175]{dahlin}. 

\subsection{Att kompensera för LCA:s svagheter}

\subsubsection{Tidsaspekten}

En LCA är som sagt ytterst tidskrävande. Strukturen är relativt rigid och lämpar sig inte i alla lägen. En viss nivå av felxibilitet kan ändå åstadkommas genom att anpassa omfattningen av en LCA. Det finns vissa föredefinierade omfattningsnivåer:  screening, complete, organisational för bakåtblickande; och scenario och region/sektor för framåtsyftande \citetwo{Wendin}. Genom att till exempel välja att göra en screening av en befintlig produkt istället för en LCA med complete-omfattning kan man bli klar snabbare och få värdefull information att agera på, men på bekostnad av exakthet och möjlighet att kommunicera resultaten utåt.

\subsubsection{Olika metoder som ger olika resultat}

Anna-Karin Jörnbrink tog i sin föreläsning upp att om en organisation är först med att göra en LCA som inkluderar en viss produkt eller produktkategori, sätter de en \emph{de facto}-standard kring vilka data som är relevanta, hur de ska samlas in och vilka uppskattningar som är rimliga. Anledningen är att senare LCA:er ska vara jämförbara med de befintliga. Därför kommer nästa LCA efterlikna den första, o.s.v. Detta kan ses som ett sätt att handskas med att olika LCA:er kan ge olika resultat, beroende på hur de utförs. Det är dock problematiskt att den första iterationen i praktiken rikserar att huggas i sten, eftersom det rimligen borde vara den första LCA:n som hade sämst förutsättningar: den är gjord utan tidigare erfarenheter eller diskussioner på området. Det kan dessutom vara så att denna första LCA i väl stor utsträckning är anpassad till den första organsisationens föreställningar och förutsättningar. 

Ett sätt att undvika denna spelteoretiska lose-lose-situation är att LCA:er utförs av en tredje part, som oberoende miljöorganisationer eller samarbetsorganisationer. Dessa kan utveckla den första LCA:n, eller driva konventionen åt ett håll som ligger i alla andra organisationers intresse, är mer objektiv, eller på annat sätt innebär en förbättring.

\subsubsection{Datakvalitet}

Problemet med bristande datakvalitet kan i viss mån kringgås genom att arbeta med intervaller. Om vi till exempel gör en LCA för tidningsläsning och vill räkna med möjligheten att läsa tidningen på Ipad, behöver vi uppskatta hur stor del av tiden som Ipaden används till detta. Här kan det vara bra att lägga in flera möjliga siffror: som minst gissar vi 3 procent, som mest 40 procent, och ''buest guess'' 19 procent. Eftersom LCA-modeller i regel är linjära bör ändring mellan fasta värden i modellen ge begripliga utslag i slutresultatet. Att underhålla en lista på intervall som kan ersätta fasta värden i modellen ger möjlighet att ''leka'' i efterhand, och se hur olika antaganden påverkar utfallet. Risken med detta är förstås att små ändringar i uppskattningar ger kraftiga utslag i slutresultatet, vilket leder till att det inte går att dra särskilt trovärdiga slutsatser. Å andra sidan har man då identifierat en intressant ''hot spot'' som det kan vara extra intressant att arbeta vidare med \citetwo[s. 28]{Wendin}.

\subsection{Alternativa metoder}

I rapporten ''Att mäta produkters miljöbelastning'', lyfts miljöpåverkansanalyser som ett kompletterande medel för att bedöma produkters miljöbelastning \citetwo[s. 15]{nat}. En sådan är en kvalitativ analys, till skillnad från en mer kvantitativ LCA, ämnad att identifiera vilka faktorer som är de viktigaste för att reducera en produkts miljöpåverkan \citetwo[s. 2-3]{eea}, och kan vara ett användbart verktyg för att identfiera områden för att göra detaljerade LCA:er \citetwo[s. 15]{nat}.

% \subsection{Ett alternativ: ABCD-modellen}

% I de lägen då en organisation inte bara vill identifiera hur stor miljöpåverkan olika produkter och tjänster har, utan också ställa om sin verksamhet till mer hållbarhet, även ur social och ekonomisk synpunkt, kan ABCD-modellen användas. Denna tar, enligt upphovsorganisationen \citetwo[s. 14]{TNS}och min egen erfarenhet, ett större perspektiv på hållbarhet, och är har dessutom som explicit syfte att sätta upp en handlingsplan för att uppnå hållbarhetsvisioner. För organisationer vilka vill förbättra sitt hållbarhetsarbete och bibehålla eller öka sina vinster kan detta vara en bättre metod än att genomföra och utvärdera en LCA.

\bibliographytwo{hemtenta2}
\bibliographystyletwo{apacite}

\clearpage
\section{En hållbarare transportsektor}
% 1483 ord!

Min analys är att de tre metoderna elektrifiering, automatisering och tjänstefiering är kompletterande utvecklingar mot en hållbar transportsektor. De är också intressanta för att tackla olika delproblem i transportsektorn, och har olika tidshorisont för när de kommer göra en markant skillnad. 

I Sverige orsakas merparten av transportsektorns koldioxidutsläpp för transporter av vägtrafik, enligt Trafikverkets siffror \citethree{tv}. Alltså skulle omställningnar i för landtransporter kunna minksa de totala utsläppen i sektorn kraftigt.

\subsection{Elektrifiering}

Elektrifiering är på många sätt den minst komplicerade utvecklingen mot en hållbar transportsektor. I teorin skulle transportsektorn kunna vara så gott som klimatneutral, borträknat tillverkning och underhåll av fordon, om den drevs av förnybara och klimatneutrala energikällor. Eftersom el, i teorin, kan framställas på klimatneutrala sätt, är därför eldrivna fordon ett på många vis självklart sätt att åstadkomma hållbar utveckling inom transportsektorn.

Här stöter vi dock det första problemet med elektrifieringen, nämligen hur ren själva elen är. I sin föreläsning (slide 8) tog Per-Erik Holmberg upp kostnaden i koldioxidutsläpp per kilometer för olika former av bilar, och det är inte självklart att elbilen står som vinnare. De större utsläppen vid tillverkning, samt de utsläpp som följer av att elen som förbrukas kommer from delvis ''smutsig'' energimix gör att elektrifiering kan leda till större utsläpp till exempel dieselbilar som drivs av biodiesel. En förutsättning för att elektrifiering ska göra transportsektorn mer hållbar är därför en mer storskalig omställning av ett lands energiförsörjning, vilket kan kräva en hel del politisk styrning, eller lönsingar där el- och bilbolag går ihop för att skapa den nödvändiga omställningen.

Ytterligare en aspekt av elektrifieringen som togs upp och som kräver samarbeten långt utanför bil\-företagens gränser är elleveransen. En förutsättning för storskalig omställning är möjligheten att ladda elfordon. Det kräver laddstolpar utspridda över hela landet. Det vore inte lönsamt för enskilda biltillverkare att bygga ut sådana, eftersom även andra tillverkare då skulle kunna dra nytta av dem utan att behöva betala för eller driva utbyggnaden. Därför skulle det krävas storskaliga samarbeten mellan biltillverkare, eller statliga eller överstatliga projekt, för att bygga ut sådana nät. På samma sätt skulle det krävas kostsamma investeringar för att bygga laddinfrastruktur för tunga fordon, till exempel genom elvägar.

\subsubsection{Rekyleffekter}

Det finns här också en överhängande risk för en rekyleffekt. Om till exempel politiska inictament, såsom högre bensinskatt, utsläppsrätter, eller något annat som driver upp priserna på transport, kan en elektrifiering leda till ökad bilkörning när det slår igenom om det då är avsevärt billigare än bensin- och dieselbilar. Detsamma gäller om en sådan prisökning drivits fram av ökade oljepriser, eller om laddning blir väldigt billigt i förhållande till bensin- och dieselpriserna.

\subsection{Automatisering}

Autonoma fordon har en klar möjlighet att köra mer miljövänligt än människor, eftersom de kan programeras att köra miljövänligt, till skillnad från mänskliga förare vilka kan ha andra intressen (som att komma fram fort), bli trötta och så vidare. Exakt hur miljövänliga sådana bilar skulle vara är dock svårt att säga. De finns däremot förutsättningar för autonoma fordon att bidra till tjänstefieringen av transportsektorn, vilket vi återkommer till.

Ur de andra hållbarhetsaspekterna är autonoma fordon desto mer intressanta. De bidrar till ekonomisk hållbarhet genom att ersätta ett dyrt men i dagsläget nödvändigt yrke, chaufför, med ett datorprogram som är i princip gratis i drift. På så vis kan både denna och framtida generationer dra nytta av billigare och snabbare transporter.

Ur det sociala perspektivet är de autonoma fordonen mer koplicerade. Enligt SCB var lastbilsförare 2014 det 16:e vanligaste yrket i Sverige med dryga 55\ 000 yrkesverksamma \citethree{scb}. En omställning till autonoma fordon skulle innebära en plötsligt ökad arbetslöshet i den här gruppen. För många av dessa kan detta komma att leda till ett socialt utanförskap, eller att de blir en extra belastning på omgivningen. Detta ställer krav på det offentliga samhället på ett sätt som inte är socialt hållbart, eftersom det inte tillgodoser dagens sociala behov, och i värsta fall gör det svårare för framtida generationer att tillfredställa sina, om utanförskapet får allvarliga bieffekter på samhället.

\subsection{Tjänstefiering}

Till skillnad från de två tidigare möjliga utvecklingarna är tjänstefiering inte något som förutsätter teknisk utveckling – det finns inga nya tekniker som måste förfinas innan den här utvecklingen blir möjlig i stor skala. De nödvändiga teknikerna - internet och enkel tillgång till internet även på språng – finns på plats, och det som återstår är att åstadkomma disruptiva tjänster och affärsmodeller som på allvar kan ändra konsumtionsmönster. Däremot kan autonoma fordon kraftigt öka tillgängligheten av flexibla transporttjänster. När en chaufför inte behöver avlönas och det inte kostar något för en bil att stå still kan priserna pressas.

Fördelen med tjänstefiering är att färre fordon kan användas till att transportera fler människor längre, genom samägda fordon och samåkning. Det finns här också möjlighet att få fler att välja mer miljövänliga färdsätt, som cykel och tåg: om man vet att det går att få tag i en bil behöver man inte alltid ha en egen med sig, och kan istället välja färdmedel som är bättre för miljön åt ena hållet, och ta bilen andra, istället för att tvunget alltid ta bilen eftersom man kan komma att behöva den. Även system för att hyra cykel istället för att äga kan ha samma effekt.

Ytterligare en fördel med tjänstefieringen är en ökad social hållbarhet. Sådana tjänster kan demokratisera resande och göra det mer flexibelt. Människor som är idag inte kan resa annat än kollektivt på egen hand kan genom smarta tjänster få tillgång till platser de tidigare behövt rationalisera bort, vilket gör att de kan ta sig till arbeten och aktiviteter på fler ställen utan att äga en egen bil. Det ökar också möjligheten till nöjesresande vilket kan ge en ökad livskvalitet för många.

\subsubsection{Rekyleffekter}

Tjänstefiering har hög risk för rekyleffekter. Om bilar blir mer tillgängliga även för de som inte idag äger en bil, till exempel genom bilpoolar och samåkningstjänster, ser jag det som att det finns en markant risk för att bilåkningen kan öka markant.

En annan aspekt som sannolikt leder till ökad bilåkning är att företagen som skapar tjänsterna har ett intresse av att konsumenter använder tjänsten, om affärsmodellen är sådan att man betalar för hur mycket man använder tjänsten. Därmed ligger det i företagens intresse att göra tjänsten tillgänglig, flexibel och attraktiv, så att konsumenter ska välja den före andra transportmedel. De tjänster som kommer vara förskonade från den effekten är de som använder en affärsmodell med fasta priser, så att de istället tjänar på att folk köper tjänsten men inte använder den. Det kan till exempel vara bilpooler med fasta månadskostnader.

\subsection{Slutsats}
% elektrifiering är inkrementell
% automatisering och tjänstefiiering är disruptiva

Elektrifiering har stor potential att göra transportsektorn mer hållbar ur ett miljöperspektiv, eftersom det finns ett visst transportbehov i den moderna ekonomin – varor och människor behöver förflyttas – och det finns ingen grad av effektivisering genom smartare körning och bättre logisitik som helt kan kompensera för detta. I slutändan måste massa förflyttas, vilket innebär arbete. Eftersom det är en sektor som i grund och botten utför fysikaliskt arbete måste den kunna utnyttja hållbara energikällor. Dessutom har FN som mål att kraftigt öka mängden hållbar energi i den globala energimixen till 2030 \citethree{un}. Om transportsektorn rör sig mot elektrifiering på vägen kan detta verka som en morot för att skynda på arbetet med att nå det målet, och vice versa. På så vis kan dessa två utvecklingar stärka varandra. Detta kan underlätta utbyggnaden av elnät för att försörja transportsektorn när elektricitet och inte drivmedel blir den huvudsakliga energibäraren. Den största nackdelen med elektrifieringen är dess brist på lönsamhet, så länge fossila bränslen är förhållandevis billiga.

Tjänstefiering har också en stor potential, men risken för allvarliga rekyleffekter är stor, då resande i personbil blir mer tillgängligt. Smarta tjänster kan fylla ett tomrum mellan dagens kollektivtrafik, i vilken stora mängder potentiella bilister reser i ett enda fordon, genom att få in fler människor i färre småfordon, men det är fortfarande ett miljömässigt mindre hållbart resande än kollektivtrafik. En övergång till eldrivna fordon kombinerat med renare elproduktion kan dock eliminera skadorna med ökat resande. Då kan tjänstefieringen verkligen komma till sn rätt, då den minskar antalet fordon som behövs, och bidrar till en ökad social hållbarhet.

Automatisering tror jag inte har mycket att bidra med i termer av att göra utvecklingen mer hållbar. Däremot så har den stor potential att skapa värde och ekonomisk utveckling. En förutsättning för att denna utveckling ska vara hållbar är dock att det går att transportera människor och varor effektivt, att detta inte leder till alltför stora utsläpp, och att samhället har en förmåga att hantera den arbetslöshet som följer på automation, för att undvika koncentration av tillgångar hos de som äger produktionsmedlen samtidigt som de som tidigare arbetade med dem saknar egen försörjning.

\paragraph{Kort angående andra transportmedel}
Jag har avgränsat min analys till landtransporter, eftersom, så vitt jag kan utröna, varken elektrifiering, automatisering eller tjänstefiering är på väg att omvälva flyg- och båttransporter, varken för person- eller varutransport.

% 1483 ord!

\bibliographythree{hemtenta3}
\bibliographystylethree{apacite}
\clearpage
\section{Flygtrafik: Politisk styrning och det hållbara samhället}

% 1205 ord!
\subsection{Ekonomiska styrmedel}

En undersökning gjord av OECD visade att utsläppsrätter, tillsmmans med skatter, verkar vara de  sätt att minska utsläpp som kostar samhället minst (''net cost to society'') \citefour[s. 11-13]{oecd}, vilket de menar ligger i linje med rådande ekonomisk teorier om hur utsläpp bäst ska minskas: genom att göra utsläpp olönsamma. Extra effektivt påpekar OECD att det är med åtgärder som höjer kostnaden på externaliteten så direkt som möjligt, det vill säga genom att göra själva externaliteten dyr att orsaka, inte genom att göra aktiviteterna som idag orsakar externaliteten dyr \citefour[s. 12]{oecd}.

Eftersom flyget idag är undantaget koldioxidskatter \citefour{dn} och det är att betrakta som en mycket effektiv form av politsk styrning för att minska utsläpp finns det stor potential att minska utsläppen med skatter. Att utvidga systemet med utsläppsrätter till fler länder eller minska tillgången på utsläppsrätter är ännu  ett alternativ. En förutsättning är dock att sådana skatter införs i många länder, till exempel på EU-nivå, eller allra helst på global nivå. Om inte kommer de länder som inför skatten i ett konkurrensmässigt underläge med länder som inte har den, menar Dahlin, vilket kan hämma den ekonomiska tillväxten i länder som inför skatten \citefour[s. 200]{dahlin}. Dahlin menar också att det finns olika syn på koldioxidskatter i olika länder – särskilt mellan de stora industrialiserade ekonomierna och de snabbväxande tidigare utvecklingsländerna, såsom Kina och Brasilien \citefour[s. 200]{dahlin}. De snabbast växande ekonomierna skulle med en platt skatt på koldioxid få bära en stor del av bördan, vilket de mycket väl skulle kunna tycka är orättvist, och inte gå med på ekonomisk styrning där inte de mest industialiserade länderna – som har de bästa förutsättningarna att minska sina utsläpp – inte betalar relativt mer per utsläppsenhet.

Ytterligare en aspekt som gör det svårt att åstadkomma globala överenskommelser med ekonomiska styrmedel kring koldioxidutsläpp är enligt Dahlin de länder och aktörer som i hög grad försörjer sig på försäljning av fossila bränslen \citefour[s. 201]{dahlin}. Dessa kan motarbeta globala överenskommelser, särskilt eftersom deras inflytande på övriga länder är oproportornerligt stort, tack vare världsekonomins beroende av petroleumprodukter.

Slutligen är det viktigt att tänka på att om flyg blir dyrare kommer de som är minst bemedlade drabbas hårdast, genom minskad rörlighet, med minskad livskvalitet som följd. Det skulle därför vara värt att undersöka möjligheten att anpassa koldioxidskatter även till konsumentens inkomst, även om det kanske är en smula utopiskt.


Trots dessa intressekonflikter finns det goda skäl att försöka öka kostnaderna för flyg med hjälp av marknadsbaserade incitament, MBI:er, enligt ett förarbete av IMF inför en rapport till G20 \citefour[s. 56]{imf}. En förutsättning menar IMF är att man först studerar hur detaljerna ska utformas och hur utvecklingsländer ska kompenseras för den relativt höga bördan de skulle få bära.

I slutändan ser jag det som att den bästa politiska styrningen för att minska flygets utsläpp är makrnadsbaserade incitament: koldioxidskatter och utsläppsrätter. Globaliseringen gör visserligen att de länder som väljer att inte vara en del av styrningen får möjlighet att attrahera flygtrafik, men jag tror i regel att människor flyger till platser som de vill besöka eller behöver göra affärer i, inte att de väljer att flyga till en plats enbart på grundval av att det är billigt. Jag tror inte att människor flyger för flygandets skull. Dessutom finns det möjlighet för till exempel EU att kräva att flygbolag köper utsläppsrätter för att få traffikera medlemsländerna, oavsett var bolagen är registrerade.

\subsection{Tekniskstyrmedel och subventioner}

OECD pekar ut subventioner som ett synnerligen kostsamt politiskt styrmedel \citefour[s. 12]{oecd}. Andersson nämner i sin föreläsning att teknikstyrmedel kan vara nödvändiga för att frambringa tekniker som det inte är lönsamt att utveckla, men som har stor potential att bidra till ökad hållbarhet \citefour{pp}. Tyvärr är det antagligen inte troligt att det kommer gå att utveckla flygplan som är tillräckligt miljövänliga för att världens befolkning ska kunna flyga mycket och hållbart.

\subsection{Reglering}

Det är också möjligt att reglera flyget, genom att till exemepl sätta ett hårt tak på hur mycket flygtrafik som tillåts i Sverige eller EU. Detta skulle kunna driva upp priset på flygtrafik genom en enkel begränsning i tillgång. Detta skulle dock kunna orsaka stora problem genom en begränsing av möjliga affärsresor till regionen som inför regeleringen, och minskad turism. Det är också en välidgt oflexibel lösning jämfört med ekonomiska styrmedel.

\subsection{Samhällsplanering}

Att flyga är det snabbaste och ibland billigaste alternativet för att resa långa sträckor. Vid vilka avstånd flyg blir en god idé från konsumentens synpunkt beror dock på andra tillgängliga alternativ som kan bidra med samma funktionella enhet: att ta någon från A till B. Storskaliga utbyggnader av till exempel järnvägsnätet till höghastighetsbanor har därmed potential att flytta viss flygtrafik till tågtrafik. Ett förslag som ''Europakorridoren'' \citefour{korr} skulle därmed kunna till viss del avlasta flygtrafiken i Sverige, och liknande projekt på Europeisk nivå skulle kunna minska behovet ännu mer, men självklart inte helt.

\subsection{Omställning till hållbar utveckling}
Vem som är ansvarig för omställning till hållbar utveckling är en på många sätt moralisk fråga, men också en praktisk sådan: vem har \emph{störst möjlighet} att bidra till en omställning? I marknadsekonomier går det att hävda att det är individerna eftersom de som har möjlighet att bilda och utveckla företag, eller ''rösta med sina pengar'' på företag de vill stötta, genom att inte konsumera sådant de anser skadligt. Tyvärr skapar detta ett ''allmännigens dilemma'', eller ''tragedy of the commons'' på engelska\footnote{För en genomgång av idén med allmänningens dilemma, se \citefour{trag}}, där den enskilde vinner på att bortse från allas bästa, men att när alla gör så blir resultatet att alla förlorar på det. Därför är det också viktigt att det finns någon gemensam styrning som skyddar samhället från den sortens kortsiktighet. Här har politikerna ett stort ansvar, eftersom de är utsedda att verka för allmänhetens bästa, och kan instifta lagar och regler som leder till att våra gemensamma resurser används klokt. Tyvärr lyder även politiker under ett allmänningens dilemma i den globala miljöfrågan, etersom de endast har direkt inflytande över en nation, och de nationer som väljer att bortse från miljöfrågan kan göra en ekonomisk vinning på att locka till sig företag och smutsiga industrier. Just därför har överstatliga organ, som EU och FN, en viktig roll att spela, i och med att de kan påverka många länder att anta lagar och regler som tvingar fram hållbar utveckling multilateralt. För att det ska lyckas måste dock det finnas en stark opinion för den sortens insatser, som sätter press på de politiker som ska ratificera besluten. Här har opinionsbildare, till exempel miljöorganisationer, en viktig roll att spela, och likaså forskare som kan uttala sig som oberoende experter, i och med att de kan informera allmänheten om hur vi kan uppnå hållbar utveckling, och varför det är nödvändigt.

Alla dessa aktörer ingår som synes i ett system med återkopplingar. Om vi antar att en del av dessa är icke-linjära, till exempel opinioner som kan röra sig fort på grund av små förändringar, har vi att göra med ett komplext system såsom Dahlin definierar det \citefour[s. 47]{dahlin}. Det är därmed mycket svårt att lägga hela ansvaret på en aktör, eftersom systemet är oförutsägbart och i praktiken omöjligt att styra från en punkt. Det är därför i någon mening allas ansvar att bidra till en hållbar utveckling.

\bibliographyfour{hemtenta4}
\bibliographystylefour{apacite}

\end{document}
