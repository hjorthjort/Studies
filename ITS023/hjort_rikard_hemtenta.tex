%General
\documentclass{article}
\usepackage[utf8]{inputenc}
\usepackage{fullpage}
\usepackage[swedish]{babel}
\selectlanguage{swedish}

% Supress warning of hboxes in bibliography.
\usepackage{etoolbox}
\apptocmd{\sloppy}{\hbadness 2057\relax}{}{}
\apptocmd{\sloppy}{\hbadness 10000\relax}{}{}

% Bibliography
\usepackage{apacite}
\usepackage{multibib}
\newcites{one}{Referenser}
\newcites{two}{Referenser}
\newcites{three}{Referenser}
\newcites{four}{Referenser}

%Symbols
\usepackage{commath}
\usepackage{amsmath}
\usepackage{amssymb}

%Numbering
\usepackage{chngcntr}
\counterwithin{figure}{section}

%Formatting
\usepackage{bussproofs}
\usepackage{amsthm}
\usepackage{mathtools}
\usepackage{graphicx}
\graphicspath{ {img/} }
\usepackage{caption}
\usepackage{hyperref}
\hypersetup{breaklinks=true}

\title{ITS023 Hemtentamen 2016}
\date{\today}
\author{Rikard Hjort} 
\begin{document}
\maketitle

\section{Bekämpa denguefeber söder om Sahara}

\subsection{Problemläget}

Det är inte helt klart hur vanligt denguefeber är i Afrika \citeone[s. 8]{WHO:2009}. Redan detta är ett stort hinder för arbetet med bekämpning, eftersom man inte vet var insatser behövs mest, vilket gör att resurser inte används så effektivt som möjligt; utbildning och stora dräneringsprojekt kan vara kostsamma insatser. Denguefeber framstår inte heller som ett lika allvarligt problem som som vanligare och dödligare sjukdomar som HIV/AIDS \citeone[s. 8]{WHO:2009}. Detta gör det svårare att få stöd till insatser och att övertyga människor i riskområden att anpassa sina vanor för att minska spridningen. De åtgärder som krävs på individnivå för att förstöra habitat för myggor, främst \emph{Aedis aegypti} \citeone[s. 60-161]{WHO:2009}, kommer vara svåra att få till om inte gemene man ser problemet som prioriterat.

Det är också svårt att övertyga människor om vikten av att bekämpa denguefeber om de inte vet hur vanligt det är, var det håller på att spridas, etc. Därför behöver informationsbristen lösas för att kunna bedriva effektivt påverkansarbete.

\subsection{Åtgärder}

Jag föreslår att vi bygger två IT-plattformar: en för att samla och en för att sprida information. Plattformarna görs tillgängliga via API:er, och betrodda aktörer kan uppdatera datan i plattformarna. På så sätt kan man göra plattformarna tillgängliga på det sätt som passar det lokala kontextet: SMS-tjänst, app, webbapplikation eller datorprogram.

\subsubsection{Samla information}

Den första plattformen (Plattform 1) har databaser för betrodda aktörer: en för att registrera bekräftade och misstänkta sjukdomsfall; en annan registrerar vaccinationer; ytterligare en innehåller platsdata om områden som haft förhöjd risk, platser där stillastående vattensamlingar bildas, och information om städer, till exempel hur sophanteringen fungerar. 

Målet med Plattform 1 är att hjälpa WHO, myndigheter och hälsovårds\-apparaten att fatta informerade beslut om var insatser behövs – särskilt vid plötsliga utbrott, då det är för sent att börja samla information, men den är som mest nödvändig. Plattform 1 skulle också kunna användas till forskning om denguefebers spridning, vilket skulle underlätta hjälparbetet ytterligare. Slutligen skulle Plattform 1 också kunna användas för att informera allmänheten, vilket vi återkommer till.

\subsubsection{Sprida information}

Den andra plattformens (Plattform 2) data görs öppet tillgänglig. Kärnan i den är en databas med de bästa råden för hur man förstör potentiella habitat för \emph{Aedis aegypti}. Till exempel föreslår WHO en rad specifika insatser \citeone[s. 60-64]{WHO:2009}. Plattformen kan utnyttja aktuell väderdata och rapporter om dengueutbrott, till exempel för automatiska utskick eller push-notifikationer till berörda parter (hushåll, arbetsplatser, myndigheter, etc.) när riskerna ökar.

Ovanpå plattformen byggs sedan applikationer till smarttelefoner, hemsidor, och SMS-tjänster som lokala regeringar och hjälporganisationer kan använda för att skicka ut information när den behövs, dit den behövs. Efter kraftiga regnfall kan man till exempel uppmana befolkningen att åtgärda vattensamlingar, till exempel med SMS, eller med videoklipp som i Singapore \citeone{Singapore:video}.
 
Plattform 2 kan också läsa data från Plattform 1 för att upptäcka ökade risker. Ett bra exempel är Singapores karta över ''aktiva kluster'' \citeone{Singapore:clusters} och övervakningsdata från Puerto Rico \citeone[s. 73]{WHO:2009}. Plattform 1 kan på så sätt ha en viktig påverkan på medvetenheten om denguefeber, vilket är högprioriterat, vilket gör att jag betraktar en Plattform 1 som den högst prioriterad.

\subsection{Följdproblem av åtgärder}

% Följdproblem %
Eftersom vi arbetar med en mycket stor region, i vilken det finns länder med många olika regimtyper och statsapparater, finns risken att sådana plattformar används för övervakning och desinformation. Om en stat till exempel använder vår plattform för att få reda på sina medborgares sjukhusbesök, eller sprida falsk data i skrämselsyfte, kan förtroendet för båda WHO och för statlig hälsoinformation kraftigt undermineras, vilket gör datainsamling och informationsspridning mycket svårt.

\subsection{Osäkerhetsparametrar}

Den höga korruptionsgraden i vissa länger, till exempel Angola \citeone{corruption:angola}, gör att det kan vara svårt och riskabelt att få med stater i vårt datainsamlingsarbete, eftersom vi inte vet om de ser Plattform 1 som en lämplig väg att sprida desinformation, eller om de alls vill delta. Jag föreslår därför att vi börjar använda Plattform 1 i ett utvalda stater och gradvis expanderar till fler, efter utvärdering.

\subsection{Övriga aspekter}

\subsubsection{Malaria}
% Lokalt kontext %
Malaria är en sjukdom som ofta överskuggar denguefeber \citeone[s. 8]{WHO:2009}. I samhällen där malaria är vanligt bör kampanjer ta hänsyn till detta. Det kan vara väldigt olämpligt att fokusera på denguefeberprevention och bättre att fokusera på malariaprevention, eftersom även malaria orsakas av myggor. Genom att anpassa sig till det lokala kontextet och i första hand fokusera på malariaprevention kan informationen – hur man förstör habitat för myggor – få maximalt genomslag.

\subsubsection{Samarbete med lokala myndigheter}
% Intressenter %
Vår grupp har som mål att utrota denguefeber. Det går dock inte att bortse från att lokala myndigheter kan prioritera andra sjukdomar högre, och inte vill lägga resurser på att använda våra lösningar. För att nå våra mål bör vi därför erbjuda oss att utbilda medicinsk personal i datainsamling. Vi behöver också göra det mycket enkelt för lokala myndigheter att bidra, genom att bygga integrationer med deras nuvarande sjukvårdssystem. Det kan till exempel röra sig om att vi extraherar data från inscannade papper från institutioner som främst använder papper.

% Inflytandesfär %
Som överstatligt organ är vi begränsade i vad vi kan göra, och vår roll bör vara att skapa förutsättningar för och bedriva påverkansarbete mot de lokala aktörerna. Vi har antagligen inte resurser att på egen hand undersöka och ta prover för att avgöra förekomsten av denguefeber. Det är något vi måste göra i samarbete med lokala myndigheter, genom att få utbilda medicinsk personal i lämplig provtagning och datainsamling, och i hur man kan använda de IT-verktyg vi tar fram.

% Bristande information
Vi vet inte än hur lokala aktörer skulle ställa sig till plattformarna, men det är de som förväntas använda dem. Jag föreslår att vi sätter ihop arbetsgrupper med representanter för WHO, lokala myndigheter, medicinsk personal och andra intressenter, för att få återkoppling på hur plattformarna bör se ut, och om de ens ska byggas. Innan dess bör vi inte påbörja någon utveckling, för att inte riskera att bygga fel saker.

\bibliographyone{hemtenta} 

\bibliographystyleone{apacite}


\clearpage
\section{Livscykelanalys}

\subsection{Fördelar}

Nästan all mänsklig nytta kan genereras på mer än ett sätt. En anteckning kan till exempel göras med en enkel blyertspenna eller reservoarpenna, men en resrvoarpenna håller längre. De olika alternativen har dessutom inte bara olika stor miljöpåverkan, utan också olika sorters miljöpåverkan: där en reservoarpenna kräver smådelar av plast och metall, bläck och mycket energi, kräver en enkel blyertspenna trä och grafit. 

Den stora fördelen med LCA är möjligheten att mäta båda olika sorters miljöpåverkan och storleken på miljöpåverkan i förhållande till nyttan som varan eller tjänsten skapar, mätt i ''funktionell enhet''. LCA tar dessutom hänsyn till hela kedjan av nyttoproduktion, ''från vaggan till graven'', så att man i siffor kan se den totala miljöpåverkan man orsakar om man väljer ett viss alternativ för nyttoproduktion istället för ett annat, och även identifiera vilka delar i processen det finns störst potential till förbättring. \citetwo[s. 95]{la} \citetwo[s. 21]{hitch}.

En annan fördel med LCA är att det finns standarder och konventioner kring hur LCA bör utföras, vilket skiljer det från verktyg som är interna i ett företag eller organisation på så sätt att fler människor kan förstå och granska en given LCA, vilket åtminstone i teorin innebär att det ställs höga krav på objektivitet i en LCA. Den allmänt accepterade strukturen för en LCA tillsammans med en gemensam terminologi underlättar diskussionen kring LCA \citetwo[s. 95]{la}. Det gör också att LCA kan användas vid inköp där köparen anser att miljöpåverkan är en viktig aspekt \citetwo[s. 176]{dahlin}.

\subsection{Nackdelar}

Lindfors och Antonsson nämner en rad nackdelar med LCA. Till exempel säger de att LCA ger en ''översikts\-bild, som alltid innehåller osäkerheter'', att de många olika metoder som går under namnet LCA gör att olika LCA:er kan ge olika resultat, att avsaknad av eller lågkvalitativ data ger stora osäkerheter och att även om LCA-metoden syftar till att vara så heltäckande som möjligt, så går det aldrig att täcka in alla aspekter \citetwo[s. 101]{la}.

Vidare nämnde Marcus Wendin i sin föreläsning att LCA inte tar upp lokala aspekter, såsom hur stora risker eller hur mycket miljöförstöring ett område utsätts för, utan alla skadeverkningar buntas ihop i totaler. Han nämnde också att en LCA förstås är väldigt tidskrävande eftersom den innehåller många steg, där vart och ett måste utföras noggrant. Till exempel behöver man i inventeringsanalysen ta hänsyn till \emph{alla} material- och energiflöden in i systemet, och likaså alla utflöden.

En LCA säger inte heller någonting om vad som \emph{borde} göras, utan kan bara beskriva hur stor miljö\-påverkan är eller skulle kunna vara. Inte heller tar en LCA hänsyn till aspekterna ekonomisk eller social hållbarhet. Till detta kan man istället använda livscykelkostnadsanalys respektive social livscykelanalys \citetwo[s. 175]{dahlin}. 

\subsection{Att kompensera för LCA:s svagheter}

\subsubsection{Tidsaspekten}

En LCA är som sagt ytterst tidskrävande. Strukturen är relativt rigid och lämpar sig inte i alla lägen. En viss nivå av flexibilitet kan ändå åstadkommas genom att anpassa omfattningen av en LCA. Det finns vissa föredefinierade omfattningsnivåer:  screening, complete, organisational för bakåtblickande; och scenario och region/sektor för framåtsyftande \citetwo{Wendin}. Genom att till exempel välja att göra en screening av en befintlig produkt istället för en LCA med complete-omfattning kan man bli klar snabbare och få värdefull information att agera på, men på bekostnad av exakthet och möjlighet att kommunicera resultaten utåt.

\subsubsection{Olika metoder som ger olika resultat}

Anna-Karin Jörnbrink tog i sin föreläsning upp att om en organisation är först med att göra en LCA som inkluderar en viss produkt eller produktkategori, sätter de en \emph{de facto}-standard kring vilka data som är relevanta, hur de ska samlas in och vilka uppskattningar som är rimliga. Anledningen är att senare LCA:er ska vara jämförbara med de befintliga. Därför kommer nästa LCA efterlikna den första, o.s.v. Detta kan ses som ett sätt att handskas med att olika LCA:er kan ge olika resultat, beroende på hur de utförs. Det är dock problematiskt att den första iterationen i praktiken riskerar att huggas i sten, eftersom det rimligen borde vara den första LCA:n som hade sämst förutsättningar: den är gjord utan tidigare erfarenheter eller diskussioner på området. Det kan dessutom vara så att denna första LCA i väl stor utsträckning är anpassad till den första organisationens föreställningar och förutsättningar. 

Ett sätt att undvika denna spelteoretiska lose-lose-situation är att LCA:er utförs av en tredje part, som oberoende miljöorganisationer eller samarbetsorganisationer. Dessa kan utveckla den första LCA:n, eller driva konventionen åt ett håll som ligger i alla andra organisationers intresse, är mer objektiv, eller på annat sätt innebär en förbättring.

\subsubsection{Datakvalitet}

Problemet med bristande datakvalitet kan i viss mån kringgås genom att arbeta med intervaller. Om vi till exempel gör en LCA för tidningsläsning och vill räkna med möjligheten att läsa tidningen på Ipad, behöver vi uppskatta hur stor del av tiden som Ipaden används till detta. Här kan det vara bra att lägga in flera möjliga siffror: som minst gissar vi 3 procent, som mest 40 procent, och ''buest guess'' 19 procent. Eftersom LCA-modeller i regel är linjära bör ändring mellan fasta värden i modellen ge begripliga utslag i slutresultatet. Att underhålla en lista på intervall som kan ersätta fasta värden i modellen ger möjlighet att ''leka'' i efterhand, och se hur olika antaganden påverkar utfallet. Risken med detta är förstås att små ändringar i uppskattningar ger kraftiga utslag i slutresultatet, vilket leder till att det inte går att dra särskilt trovärdiga slutsatser. Å andra sidan har man då identifierat en intressant ''hot spot'' som det kan vara extra intressant att arbeta vidare med \citetwo[s. 28]{Wendin}.

\subsection{Alternativa metoder}

I rapporten ''Att mäta produkters miljöbelastning'', lyfts miljöpåverkansanalyser som ett kompletterande medel för att bedöma produkters miljöbelastning \citetwo[s. 15]{nat}. En sådan är en kvalitativ analys, till skillnad från en mer kvantitativ LCA, ämnad att identifiera vilka faktorer som är de viktigaste för att reducera en produkts miljöpåverkan \citetwo[s. 2-3]{eea}, och kan vara ett användbart verktyg för att identifiera områden för att göra detaljerade LCA:er \citetwo[s. 15]{nat}.

% \subsection{Ett alternativ: ABCD-modellen}

% I de lägen då en organisation inte bara vill identifiera hur stor miljöpåverkan olika produkter och tjänster har, utan också ställa om sin verksamhet till mer hållbarhet, även ur social och ekonomisk synpunkt, kan ABCD-modellen användas. Denna tar, enligt upphovsorganisationen \citetwo[s. 14]{TNS}och min egen erfarenhet, ett större perspektiv på hållbarhet, och är har dessutom som explicit syfte att sätta upp en handlingsplan för att uppnå hållbarhetsvisioner. För organisationer vilka vill förbättra sitt hållbarhetsarbete och bibehålla eller öka sina vinster kan detta vara en bättre metod än att genomföra och utvärdera en LCA.

\bibliographytwo{hemtenta2}
\bibliographystyletwo{apacite}

\clearpage
\section{En hållbarare transportsektor}

\subsection{Elektrifiering}

Elektrifiering är därför på många sätt den minst komplicerade utvecklingen mot en hållbar transportsektor. Om alla vägfordon drevs av el från klimatneutrala, förnybara källor skulle transportsektorns utsläppa minska kraftigt.

Här stöter vi dock det första problemet med elektrifieringen: elproduktion. I sin föreläsning (slide 8) tog Per-Erik Holmberg upp kostnaden i koldioxidutsläpp per kilometer för olika former av bilar. De större utsläppen vid tillverkning av elbilar, samt de utsläpp som följer av att elen som förbrukas kommer från en delvis icke-klimatneutral energimix gör att elektrifiering kan leda till större utsläpp än till exempel dieselbilar som drivs av biodiesel. En förutsättning för att elektrifiering ska göra transportsektorn mer hållbar är därför en mer storskalig omställning av ett lands energiförsörjning, vilket kan ta lång tid.

Ytterligare ett hinder för elektrifieringen som togs upp på föreläsningen är elleveransen. En förutsättning för storskalig omställning är möjligheten att ladda elfordon på vägarna. Det vore inte lönsamt för enskilda biltillverkare att bygga ut nätverket av laddstolpar, eftersom även andra tillverkare då skulle kunna dra nytta av dem utan att behöva betala för eller driva utbyggnaden. Därför skulle det krävas samarbeten mellan biltillverkare, eller statliga eller överstatliga projekt. På samma sätt skulle det krävas kostsamma investeringar för att bygga laddinfrastruktur för tunga fordon, till exempel genom elvägar.

\subsubsection{Rekyleffekter}

Om elbilsteknik blir billig, till exempel genom subventioner, kan bilkörning och -ägande öka proportionerligt mot kostnadsminskningen. Ett sätt att motverka detta är att istället fokusera på att göra fossila bränslen dyrare, vilket drabbar ekonomin snarare än miljön.

\subsection{Automatisering}

Det finns vissa indikationer på att autonoma fordon kan åstadkomma sotra utsläppminskningar relativt manuella fordon \citethree{aut}. Det finns också förutsättningar för autonoma fordon att bidra till tjänstefieringen av transportsektorn, vilket vi återkommer till.

Ur ekonomiskt och socialt hållbarhetsperspektiv är autonoma fordon ännu mer intressanta. De bidrar till ekonomisk hållbarhet genom att ersätta ett dyrt men i dagsläget nödvändigt yrke, chaufför, med ett datorprogram som är i princip gratis i drift. På så vis kan både denna och framtida generationer dra nytta av billigare och snabbare transporter.

Ur det sociala perspektivet är de autonoma fordonen mer komplicerade. Enligt SCB var lastbilsförare 2014 det 16:e vanligaste yrket i Sverige med dryga 55\ 000 yrkesverksamma \citethree{scb}. Plötslig massarbetslöshet bland yrkesförare kan leda till ökat socialt utanförskap och belastningar på det offentliga samhället. Detta är inte socialt hållbart, eftersom det inte tillgodoser dagens sociala behov, och i värsta fall gör det svårare för framtida generationer att tillfredsställa sina, om utanförskapet får allvarliga bieffekter på samhället.

Autonoma fordon är dessutom socialt problematiska såtillvida att de skapar moraliska dilemman, till exempel vem som är ansvarig vid en krock. Detta kan dock uppvägas av att autonoma fordon är förväntas vara mer säkra, och dessutom bättre kan utnyttja väginfrastruktur, enligt föreläsningen med Per-Erik Holmberg.

\subsubsection{Rekyleffekter}

En potentiell rekyleffekt av automatisering är förstås ökat bilåkande, eftersom det blir enklare och smidigare att åka bil när man som människa inte behöver köra den.

\subsection{Tjänstefiering}

Fördelen med tjänstefiering som den beskrevs i föreläsningen är att färre fordon kan användas till att transportera fler människor längre, genom samägda fordon och samåkning. Det finns också möjlighet att få fler att välja mer miljövänliga färdsätt, som cykel och tåg: om man vet att det går att få tag i en bil behöver man inte alltid ha en egen med sig, och kan välja miljövänliga färdsätt åt ena hållet, och bil åt andra. 

Ytterligare en fördel med tjänstefieringen är en ökad social hållbarhet. Sådana tjänster kan demokratisera resande genom att göra det billigt och flexibelt. Människor som idag inte har råd med bil kommer kunna ta sig till arbeten och aktiviteter på fler ställen. Det ökar också möjligheten till nöjesresande vilket kan ge en ökad livskvalitet för många.

Tjänstefiering är också lovande eftersom det saknas tekniska hinder: disruptiva tjänster som ändrar konsumtionsmönster är allt som behövs. Däremot kan  automatisering kraftigt öka tillgängligheten av flexibla transporttjänster, eftersom tillgången till chaufförer inte är en begränsning.

\subsubsection{Rekyleffekter}

Tjänstefiering har hög risk för rekyleffekter, eftersom de gör bilkörning mer tillgängligt, genom till exempel bilpooler och samåkningstjänster, som kan locka människor som idag reser mer miljövänligt att börja köra istället.

En annan aspekt som sannolikt leder till ökad bilåkning är att företagen som skapar tjänsterna ofta har ett intresse av att konsumenter använder tjänsten. Därmed ligger det i företagens intresse att göra tjänsten tillgänglig, flexibel och attraktiv, för att locka kunder att köra mycket. Affärsmodell med fasta priser, som tjänar på att kunder köper tjänsten men inte använder den, är ett sätt att undvika det incitamentet för företagen.

\subsection{Slutsats}

De omställningar som Per-Erik Holmberg diskuterade i sin föreläsning gällde framförallt vägtransporter. I Sverige orsakas merparten av transportsektorns koldioxidutsläpp för transporter av vägtrafik, enligt Trafikverkets siffror \citethree{tv}. Alltså skulle omställningar i vägtransporter kunna minska de totala utsläppen i sektorn kraftigt.

Elektrifiering har stor potential att vara nyckeln till en hållbar transportsektor. Eftersom det är en sektor som i grund och botten utför fysikaliskt arbete måste den kunna utnyttja hållbara energikällor. Ingen grad av effektivisering kan kompensera helt för detta. Dessutom har FN som mål att kraftigt öka mängden hållbar energi i den globala energimixen till 2030 \citethree{un}, och om transportsektorn elektrifieras ökar incitamenten för detta ytterligare. Den största nackdelen med elektrifieringen är dess brist på lönsamhet, så länge fossila bränslen är förhållandevis billiga.

Tjänstefiering har också en stor potential, men risken för allvarliga rekyleffekter är stor. En övergång till eldrivna fordon kombinerat med renare elproduktion kan dock eliminera skadorna med ökat resande, och då kan tjänstefieringen verkligen komma till sin rätt, då den minskar antalet fordon som behövs, kan effektivisera resor och bidrar till en ökad social hållbarhet.

Automatisering har stor potential att skapa ekonomisk utveckling, men förutsättning för att denna utveckling ska vara hållbar är följande: att det går att transportera människor och varor effektivt och miljövänligt, och att samhället har en förmåga att undvika koncentration förmögenhetskoncentration hos ägarna av produktionsmedlen, när robotar kör och människor blir arbetslösa.

De olika utvecklingarna kan alla bidra till hållbarhet, och kan förstås vara kompletterande.

\bibliographythree{hemtenta3}
\bibliographystylethree{apacite}

\clearpage
\section{Flygtrafik: Politisk styrning och det hållbara samhället}

% 1205 ord!
\subsection{Marknadsbaserade incitament}

En undersökning gjord av OECD visade att utsläppsrätter, tillsammans med skatter, verkar vara de  sätt att minska utsläpp med minst kostnad för samhället (''net cost to society''), vilket de menar ligger i linje med rådande ekonomisk teorier om hur utsläpp bäst ska minskas: genom att göra utsläpp marknadsekonomiskt olönsamma. Extra effektivt påpekar OECD att det är med åtgärder som höjer kostnaden på externaliteten så direkt som möjligt, det vill säga genom att göra själva externaliteten dyr, inte aktiviteterna som idag orsakar externaliteten \citefour[s. 11-13]{oecd}.

Eftersom flyget idag är undantaget koldioxidskatter \citefour{dn} vilket är att betrakta som en mycket effektiv form av politisk styrning för att minska utsläpp finns det stor potential att minska utsläppen med skatter. (Att utvidga systemet med utsläppsrätter till fler länder eller minska flygindustrins tillgång till utsläppsrätter är ännu ett alternativ.) En förutsättning är dock att sådana skatter införs i många länder, allra helst på global nivå. Om inte kommer de länder som inför skatten i ett konkurrensmässigt underläge med länder som inte har den, menar Dahlin, vilket straffar dessa länder. Dahlin menar också att det finns olika syn på koldioxidskatter i olika länder – särskilt mellan de stora industrialiserade ekonomierna och de snabbväxande tidigare utvecklingsländerna, såsom Kina och Brasilien, och att det inte finns någon självklar ''rättvis'' skatteskala \citefour[s. 200]{dahlin}. Detta gör det svårt att nå lämpliga överenskommelser.

Ytterligare ett hinder för globala överenskommelser kring ekonomiska styrmedel är enligt Dahlin de länder och aktörer som i hög grad försörjer sig på försäljning av fossila bränslen \citefour[s. 201]{dahlin}. Dessa kan motarbeta globala överenskommelser, särskilt eftersom deras inflytande på övriga länder är oproportionerligt stort på grund av världsekonomins beroende av petroleumprodukter.

Slutligen är det viktigt att tänka på att om flyg blir dyrare kommer de som är minst bemedlade drabbas hårdast, genom minskad rörlighet.

Trots dessa intressekonflikter finns det goda skäl att försöka öka kostnaderna för flyg med hjälp av marknadsbaserade incitament. Ett förarbete av IMF inför en rapport till G20 \citefour[s. 56]{imf} föreslår just detta, under förutsättning att man först studerar hur detaljerna ska utformas och hur utvecklingsländer ska kompenseras för den relativt höga bördan de skulle få bära.

Mycket talar alltså för potentialen hos marknadsbaserade incitament, som koldioxidskatter och utsläppsrätter, att göra flygtrafiken mer hållbar.

\subsection{Teknikstyrmedel och subventioner}

OECD pekar ut subventioner som ett synnerligen kostsamt politiskt styrmedel \citefour[s. 12]{oecd}. David Andersson nämner i sin föreläsning att teknikstyrmedel som statliga upphandlingar, subventioner och feed-in-tariffer kan vara nödvändiga för att frambringa tekniker som det inte är lönsamma att utveckla, men som har stor potential att bidra till ökad hållbarhet. Tyvärr är detta också väldigt riskabelt, eftersom det är omöjligt att i förväg veta vilka tekniker som kommer vara framgångsrika, och mycket pengar därför kan investeras till lite nytta \citefour{pp}.

\subsection{Reglering}

Det är också möjligt att reglera flyget, genom att till exempel endast tillåta de idag miljövänligaste flygplanen. Faran med detta är att man då låser in teknikutvecklingen och gör det svårt och oattraktivt att utveckla nya flygplan. Det är svårt att veta om de flygplanen bygger på den bästa tekniken, eller om det är möjligt att andra befintliga och möjliga tekniker kan bli mer miljövänliga, om de får förfinas. \citefour{pp}.

\subsection{Samhällsplanering}

Att flyga är det snabbaste och ibland billigaste alternativet för att resa långa sträckor. Vid vilka avstånd flyg blir en god idé från konsumentens synpunkt beror dock på vilka resealternativ som finns tillgängliga. Till exempel kan storskaliga utbyggnader av järnvägsnätet till höghastighetsbanor ha potential att flytta viss flygtrafik till tågtrafik. Ett förslag som ''Europakorridoren'' \citefour{korr} skulle därmed kunna till viss del avlasta flygtrafiken i Sverige, och till Danmark. Liknande projekt på Europeisk nivå skulle kunna minska behovet ännu mer, men självklart inte helt.

\subsection{Omställning till hållbar utveckling}
Vem som är ansvarig för omställning till hållbar utveckling är en delvis moralisk fråga, men också en praktisk: vem har \emph{störst möjlighet} att bidra till omställningen? I marknadsekonomier går det att hävda att det är individerna, eftersom de har möjlighet att bilda och utveckla företag, eller ''rösta med sina pengar'' på företag som bidrar till hållbarheten. Tyvärr skapar detta ett ''allmänningens dilemma'', eller ''tragedy of the commons'' på engelska\footnote{För en genomgång av idén med allmänningens dilemma, se \citefour{trag}}: den enskilde vinner alltid på att bortse från andras bästa, med resultatet att situationen blir sämre för alla. Därför är det också viktigt att det finns någon styrning som skyddar samhället från individens kortsiktighet. Här har politikerna en stor möjlighet, eftersom de är utsedda att verka för allmänhetens bästa, och kan instifta lagar och regler som leder till att våra gemensamma resurser används klokt. Tyvärr lyder även politiker under allmänningens dilemma i den globala miljöfrågan, eftersom de endast har direkt inflytande över en nation, och de nationer som väljer att bortse från miljöfrågan vinner på att locka till sig företag och smutsiga industrier. Just därför har överstatliga organ, som EU och FN, en viktig roll att spela, i och med att de kan påverka många länder att skapa incitament multilateralt. För att det ska lyckas måste det dock finnas en stark opinion för den sortens insatser, som sätter press på de politiker som ska ratificera besluten. Här har opinionsbildare, till exempel miljöorganisationer, en viktig roll att spela, och likaså forskare som kan uttala sig som oberoende experter, i och med att de kan informera allmänheten om hur vi kan uppnå hållbar utveckling, och varför det är nödvändigt.

Alla dessa aktörer ingår som synes i ett system med återkopplingar. Om vi antar att en del av dessa är icke-linjära, till exempel opinioner som kan röra sig fort på grund av små förändringar, har vi att göra med ett komplext system såsom Dahlin definierar det \citefour[s. 47]{dahlin}. Det är därmed mycket svårt att lägga hela ansvaret på en aktör, eftersom systemet är oförutsägbart och i praktiken omöjligt att styra från en punkt. Det är därför i någon mening allas ansvar att bidra till en hållbar utveckling, och det är nödvändigt att alla ovan nämnda aktörer i någon utsträckning bidrar.

\bibliographyfour{hemtenta4}
\bibliographystylefour{apacite}

\end{document}
