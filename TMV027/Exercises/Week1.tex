%General
\documentclass{article}
\usepackage[utf8]{inputenc}

%Symbols
\usepackage{commath}
\usepackage{amsmath}

%Formatting
\usepackage{hyperref}
\usepackage{amsthm}
\usepackage{alltt}
\newtheorem{definition}{Definition}[section]
\newtheorem{theorem}{Theorem}[section]
\newtheorem{example}{Example}[section]
\hypersetup{colorlinks=true}
\hypersetup{colorlinks=true}
\usepackage{graphicx}
\graphicspath{ {img/} }
\usepackage{caption}

\begin{document}

\section{Logic}

\begin{enumerate}
    \item
        \begin{enumerate}
            \item $ p \land q $
            \item $ p \to r $
            \item \label{itm:arrows} $ \lnot p \to \lnot q \land r \Leftrightarrow p \to q \lor \lnot r $
            \item $ q \leftrightarrow \lnot p $
            \item $ \lnot r \to q $
        \end{enumerate}
        General question: When exactly do I want to use a double arrow
        ($\Leftrightarrow$) and when do i want single arrow
        ($\leftrightarrow$). Have I used the conventions correctly in
        \ref{itm:arrows}, where I used the double arrow to denote two
        statements that are logically equivalent, but single arrow to denote
        the statement, that might be true or false?
\end{enumerate}

\end{document}
