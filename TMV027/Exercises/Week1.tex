%General
\documentclass{article}
\usepackage[utf8]{inputenc}
\usepackage{fullpage}

%Symbols
\usepackage{commath}
\usepackage{amsmath}
\usepackage{amssymb}
\usepackage{braket}

%Formatting
\usepackage{hyperref}
\usepackage{amsthm}
\usepackage{alltt}
\newtheorem{definition}{Definition}[section]
\newtheorem{theorem}{Theorem}[section]
\newtheorem{example}{Example}[section]
\hypersetup{colorlinks=true}
\hypersetup{colorlinks=true}
\usepackage{graphicx}
\graphicspath{ {img/} }
\usepackage{caption}

\begin{document}

\section{Logic}

\begin{enumerate}
    \item
        \begin{enumerate}
            \item $ p \land q $
            \item $ p \to r $
            \item \label{itm:arrows} $ \lnot p \to (\lnot q \land r) \Leftrightarrow (q \lor \lnot r) \to p $
            \item $ q \leftrightarrow \lnot p $
            \item $ \lnot r \to q $
        \end{enumerate}
        General question: When exactly do I want to use a double arrow
        ($\Leftrightarrow$) and when do i want single arrow
        ($\leftrightarrow$). Have I used the conventions correctly in
        \ref{itm:arrows}, where I used the double arrow to denote two
        statements that are logically equivalent, but single arrow to denote
        the statement, that might be true or false?
    \item
        \begin{enumerate}
            \item If it is raining and the sun is shining, then there are clouds in the sky;
            \item If when it is raining there must be clouds in the sky, then the sun is shining;
            \item It is not raining if and only if either the sun is shining or there are clouds in the sky;
            \item It is not the case that it is not raining if and only if either the sun is shining or there are clouds in the sky;
            \item Is neither raining or the sun is shining, and there are clouds in the sky.
        \end{enumerate}
    \item The weather where I am right now is cloudy, no sun, no rain. I will give truth values based on this.
        
        Excercise 1:
        \begin{enumerate}
            \item false
            \item true
            \item true
            \item true
            \item true
        \end{enumerate}

        Excercise 2:
        \begin{enumerate}
            \item true
            \item false
            \item true
            \item false
            \item true
        \end{enumerate}

    \item 
        \begin{itemize}
            \item $\lnot q \Rightarrow \lnot p$
            \item $\lnot p \lor q$
        \end{itemize}

    \item
        \begin{enumerate}
            \item 
                \begin{tabular}{c | c | c | c | c | c}
                    p & q & $ p \Rightarrow q $ & $p \lor \lnot q $ & $ p \lor q $ & Complete \\ \hline
                    T & T & T & T & T & T \\
                    T & F & F & T & T & T \\
                    F & T & T & F & T & T \\
                    F & F & T & T & F & F
                \end{tabular}
        \end{enumerate}
    \item If $p \Rightarrow q $ is false then p is true and q is false.
        \begin{itemize}
            \item $p \land q$ is false (since q is false);
            \item $p \lor q$ is true (since p is true);
            \item $q \Rightarrow p$ is true, since an implication is always true when the antecendent is false.
        \end{itemize}
    \item With logical symbols the statement is $\forall x : \exists y : y < x$. Negating yields 
        \begin{align*}
            \lnot(\forall x : \exists y : y < x) \Leftrightarrow \\
            \exists x : \lnot(\exists y : y < x) \Leftrightarrow \\
            \exists x : \forall y : \lnot (y < x) \Leftrightarrow \\
            \exists x : \forall y : y \geq x.
        \end{align*}
        which in English could be expressed as "For some number $x$ every number $y$ is such that $y \geq x$" which means there is a smallest number, that is smaller then every other.

    \item
        \begin{align*}
            \lnot \forall x.(P(x) \Rightarrow Q(x)) \Leftrightarrow \\
            \exists x.\lnot (P(x) \Rightarrow Q(x)) \Leftrightarrow \\
            \exists x. P(x) \land \lnot Q(x)
        \end{align*}
        
    \item
        $ \forall x . \exists y . \exists z . L(x,y,z)$

    \item
        $(\forall x . \exists y . F(x,y)) \land (\exists x . \forall y . F(x,y)) \land \lnot (\forall x. \forall y . F(x,y))$

    \item
        \begin{tabular}{c | c | c}
             & $ (0,1) $ & $ [0,1] $ \\ \hline
            a) & true & false \\
            b) & true & true \\
            c) & false & false \\
            d) & false & true
        \end{tabular}
\end{enumerate}

\section{Sets}

\begin{enumerate}
    \item 
        \begin{enumerate}
            \item ${a, o, u, e, i}$ (English vowels)
            \item ${12,15,18}$
            \item This cannot be enumerated, since every number $5k + 1, k \in \mathbb{N}$ is a natural number and has a remainder of 1 when divided by five. Since there are an infinite number of possible values for $k$, and each choice of $k$ renders a different number, this is not an enumeratable set. We can, however, define it using the characteristic property: 
                $$\{x \in \mathbb{N} | x \equiv 1 \mod{5}\} $$
        \end{enumerate}
    \item
        \begin{enumerate}
            \item $\{ x \in \mathbb{N} | 4 \leq x \leq 20 \land x \equiv 0 \mod{4}\}$
            \item $\{ w | w \in \Sigma^3, \Sigma = \{0,1\} \}$
            \item $\{ x^2 | x \in \mathbb{Z}_+\}$
        \end{enumerate}

    \item 
        $A \times B = \{(a,p), (a,q),(b,p),(b,q), (c,p),(c,q)\}$

        $A^2 = A \times A = \{(a,a), (a,b), (a,c), (b,a), (b,b), (b,c),(c,a), (c,b), (c,c)\}$

        $B^3 =  \{ (p,p,p), (p,p,q), (p,q,p), (p,q,q), (q,p,p),(q,p,q),(q,q,p), (q,q,q)\}$

    \item
        \begin{enumerate}
            \item First false, second (trivially) true;
            \item First true, second false (1 is not a set and can't be a subset of anything);
            \item Both true: \{1\} is an element, the second in the enumeration. The set \{1\} is also a subset, containing only 1, the first element in the enumeration;
            \item True: this is the subset containing only the second element in the enumeration;
            \item False;
            \item First true and second false: \{2\} is an element, but \{\{2\}\} is not;
            \item First false, second true.
        \end{enumerate}

    \item 
        \begin{enumerate}
            \item $\{9,11\}$;
            \item $\{8, 9, 10, 11, 12, 0, 3, 6\}$;
            \item $\{0, 2,4,6,8,10,12\}$;
            \item 
                \begin{align*}
                    \overline{B} &= \{0, 1, ..., 7\} \\
                    A \cup \overline{B} &= \{0,1,2,3,4,5,6,7,9,11\} \\
                    (A \cup \overline{B}) \cap C &= \{0, 3,6,9\}
                \end{align*}
            \item 
                \begin{align*}
                    \overline{A \cup C} \cup \overline{C} = \\
                    (\overline{A} \cap \overline{C}) \cup \overline{C} = \\
                    \overline{A} \cap (\overline{C} \cup \overline{C}) = \\
                    \overline{A} \cap \overline{C} = \\
                    \{x | x \text{ is even } \land x \not\equiv 0 \mod{3}\} = \\
                    &\{2,4,8,10\}
                \end{align*}
        \end{enumerate}
        
    \item 
        \begin{proof}

        \begin{tabular}{l | l | l}
            & Statement & Justification \\ \hline
            1 & $ \overline{\overline{A} \cap B}$ & Given \\
            2 & $ \overline{\overline{A}} \cup \overline{B}$ & de Morgan's Law \\
            3 & $ A \cup \overline{B} $ & Complements \\
        \end{tabular}

        We find that $A \cup \overline{B} $ is in our table, which was constructed using only laws of sets and the given set, and we thus conclude that the given set $\overline{\overline{A} \cup B} = A \cup \overline{B}$.
    \end{proof}
        
    \item
        \begin{enumerate}
            \item 
                \begin{proof}
                    Assume an element $x$ is in $A$ but not in $B$. Now, by definition of difference, $x$ is in  $A - B = \{a | a \in A \land a \not\in B\}$. However, since $B - A = \{b | b \in B \land b \not\in A\}$ we conclude that $x$ is not in $B - A$ (since $x$ fails both conditions) and thus $A - B \not= B - A$ in this case.
                \end{proof}

            \item
                \begin{proof}
                    We begin by defining the sets by characteristics.
                    \begin{align}
                        A - (B - C) &= \{x | x \in A \land \lnot(x \in B  \land \lnot(x \in C))\} \label{eq:1} \\
                        (A - B) - C &= \{x | x \in A \land \lnot(x \in B) \land \lnot(x \in C)\} \label{eq:2}
                    \end{align}
                    
                    We ommit parentheses in the characteristics in \ref{eq:2} since conjunction is associative.

                    We see that we have two different conditions for $x$ to be in each of the sets. If we can prove that these conditions are not logically equivalent, then we know that for some element they have a different truth value, meaning that that element is in one of the sets and not the other, which would prove that difference is not associative.

                    \begin{align}
                        &x \in A \land \lnot(x \in B  \land \lnot(x \in C)) &\\
                        \label{eq:3}\Leftrightarrow &x \in A \land (\lnot(x \in B) \lor x \in C) & \text{de Morgan}\\
                        \label{eq:4}\not\Leftrightarrow & x \in A \land \lnot(x \in B) \land \lnot(x \in C) & \\
                    \end{align}
                    \textbf{Proof by counterexample of the last ineqivalence}: Assume $x$ is in $A$ and $C$ but not in $B$. Now \ref{eq:3} is true, but \ref{eq:4} is false since the last term in the conjunction fails.
                \end{proof}
        \end{enumerate}

    \item
        \begin{enumerate}
            \item \label{proof:neg} $A - B = \{x | x \in A \land x \not\in B\}$ and $A \cap \overline{B} = \{x | x \in A \land x \in \overline{B}\}$. Since an element is not in $B$ iff it is in $\overline{B}$, these sets are equivalent.
            \item 
                \begin{proof}
                    (If part): By definition $A \subseteq B \Leftrightarrow (x \in A \Rightarrow x \in B)$. Thus we know every element in $A$ is also in B. $A - B = \{x | x \in A \land x \not\in B\}$. Since $x \in A \Rightarrow x \in B$, this means $A - B = \emptyset$.

                    (Only-if part): Conversely, when $A - B = \{x|x\in A \land x \not\in B\} = \emptyset$ we know there are no elements in A which are not also in B, since no element satisfies the condition, meaning $ x \in B \lor x \not\in A$ which is equivalent to $x \in A \Rightarrow x \in B$, which means $A \subseteq B$.
                \end{proof}
            \item 
                \begin{proof}
                    $A - (A - B) = A - (A \cap \overline B) = A \cap \overline{A \cap \overline{B}}$ as shown in \ref{proof:neg}. By de Morgan's law and laws of complement $A - (A - B) = A \cap (\overline{A} \cup \overline{\overline{B}}) =  A \cap (\overline{A} \cup B)$ and by the distributive law, law of complements and laws for $\emptyset$, $ A \cap (\overline{A} \cup B) = (A \cap \overline{A}) \cup (A \cap B) = \emptyset \cup (A \cap B) = A \cap B$ 
                \end{proof}

            \item For an element to be in $A \cap B$ it must be in both $A$ and $B$. For it to be in the superset, it must be either in $A\cap B$ or $B \cap \overline{C}$, or both. Since for every proposition $p$ it is true that $p \Rightarrow p \lor q$, if an element is in $A \cap B$ it must be in $(A \cap B) \cup (B \cap \overline{C})$, which is the definition of being a subset.
            \item The subset in this task requires that an element $x$ has the property $(x \in A \lor x \in C) \land (x \in B \lor x \not\in C)$. Any element fulfilling this must be in $B$ and $C$ or $A$ and $\overline{C}$ or in both $A$ and $B$ for both parts of the conjunction to hold. (Note that an element can't be both in $C$ and $\overline{C}$). So for every such element, either $x \in A$ holds, or $x \in B$ holds. 
            \item $A \cap B = \emptyset$ means that no element $x$ in the universe is such that $x \in A \land x \in B$ which is the same as
                \begin{align*}
                    \forall x . \lnot (x \in A \land x \in B) \Leftrightarrow \\
                \forall x . x \in \overline{A} \lor x \in \overline{B} \Leftrightarrow \\
                \forall x . x \in A x \not\in \overline{A} \Rightarrow x \in \overline{B} \text{ and vice versa }
                \end{align*}
                So, every element that is in $A$ must be in $\overline{B}$ and vice versa, which is the same as $A \subseteq \overline{B}$ and vice versa, Q.E.D.
            \item
                $$\begin{array}{r l l}
                    &A \subseteq B &\Leftrightarrow \\
                    \text{(Definition of subset)} &x \in A \Rightarrow x \in B &\Leftrightarrow\\
                    \text{(Implication)} &x \in B \lor \lnot(x \in A) &\Leftrightarrow \\
                    \text{(Double negation)} &\lnot\lnot(x \in B \lor \lnot(x \in A)) &\Leftrightarrow \\
                    \text{(de Morgan)} &\lnot(\lnot(x \in B) \land (x \in A)) & \Leftrightarrow \\
                    \text{(Definition of set complement)} &\lnot(x \in \overline{B} \land x \in A) & \Leftrightarrow \\
                    \text{(de Morgan)} &\lnot(x \in \overline{B}) \lor \lnot(x \in A) & \Leftrightarrow \\
                    \text{(Definition of set complement)} &\lnot(x \in \overline{B}) \lor x \in \overline{A} & \Leftrightarrow \\
                    \text{(Implication)} & x \in \overline{B} \Rightarrow x \in \overline{A} & \Leftrightarrow \\
                \end{array}$$
                $$\overline{B} \subseteq \overline{A}$$
            \item
            \item
            \item
        \end{enumerate}
\end{enumerate}

\section{Relations}
    
\begin{enumerate}
    \item 
        \begin{enumerate}
            \item Symmetric, transitive
            \item Antisymmetric
            \item Antisymmetric, transitive
            \item  Reflexive, symmetric, transitive (equivalence relation)\label{rel:int} 
            \item Reflexive, antisymmetric, transitive
            \item \label{rel:square} Reflexive, symmetric, transitive (equivalence relation)
        \end{enumerate}

    \item In \ref{rel:int} the class consists of all numbers with the same integer part, that is all numbers $x$ and $y$ for which $\lfloor x \rfloor = \lfloor y \rfloor$. Each class is infinite. For example, the class of 3 is all numbers which are in the semi-open interval $[3, 4)$.

        In \ref{rel:square} each class consists of two numbers, $x$ and $-x$.
    \item We need to prove that if $x R y$ is defined as $x \Leftrightarrow y$ where $x$ and $y$ are propositions, then this relation is reflexive, symmetric and transitive.

        (Reflexive) $x \Leftrightarrow x$ is true, since $x$ must always have the same truth value as itself.

        (Symmetric) $x R y$ means that $x and y$ will have the same truth value for every combination of truth values for atoms in the universe. Thus, $(x \Leftrightarrow y) \Leftrightarrow (y \Leftrightarrow x)$ and thus $x R y \Leftrightarrow y R x$.

        (Transitive) $x R y$ and $y R z$ means that $y$ and $z$ has the same truth value for every combination of truth values for all atoms in the universe. Thus, for any combination $C$ of values for every atom in the universe, $z$ and $y$ share truth values. By the first relation, $x$ has the same truth value as $y$ for the combination $C$. So for $C$ we have $x \Leftrightarrow z$. Since $C$ is arbitrarily chosen, $\forall x,z. x\Leftrightarrow z$ and thus $x R z$.

        Since $R$ is reflexive, symmetric and transitive, logical equivalence is an equivalence relation.

    \item 
        As above, we must show that $R$ is reflexive, symmetric and transitive.

        (Reflexive) $xRx$ means $x-x = 4k, k \in \mathbb{Z}$. Since $x-x = 0 = 4*0$ and $0 \in \mathbb{Z}$, $R$ is reflexive.

        (Symmetric) If $xRy$, then $x-y = 4k, k\in \mathbb{Z}$. Since $x-y=4k \Rightarrow -(x - y) = -4k \Leftrightarrow y - x = 4*(-k)$. By definiton $k \in \mathbb{Z} \Rightarrow -k \in \mathbb{Z}$ so $yRx$ and the relation is reflexive.

        (Transitive) If $xRy$ and $yRz$, then $x - y = 4k$ and $y - z = 4l$, $k,l \in \mathbb{Z}$. Thus, $y = 4l + z$ and $x - (4k + z) = 4k \Rightarrow x - z = 4k - 4l = 4(k - l)$. Since $k-l \in \mathbb{Z}$, $xRz$.

        We have shown that $R$ is reflexive, symmetric and transitive, and is thus an equivalence relation.

        $R$ partitions $\mathbb{Z}$ into 4 classes: 
        \begin{align*}
            \{0,4,8, ...\} \\
            \{1,5,9, ...\} \\
            \{2, 6, 10, ...\} \\
            \{3, 7, 11, ...\}
        \end{align*}
        This is the same as the classes of $\mathbb{Z}_4$, where $[x] = \{ y | y \equiv x \mod{4}\}$


    \item 
        We try all combinations:

        (Reflexive and symmetric) Let $A = \{a,b,c\}$ and 
        $$R_1 = \{(a,a), (b,b), (c,c), (a,b), (b,a), (b,c), (c,b)\}$$ Now every eleent in $A$ relates to itself, and every element that relates to another one has the other element relating back to it. However, $a$ relates to $b$ and $b$ relates to $c$, but $a$ does not relate to $c$, so the relation is not transitive.

        (Reflexive and transitive) Let $A$ be as before and 
        $$R_2 = \{(a,a), (a,b), (b,b), (b,c), (c,c), (a,c)\}$$
        Now every element in A relates to itself. Also, whenever there is a relationship $xRy$ and $yRz$, then also $xRz$. There is only one such nontrivial case: $aRb$ and $bRc$. As required, now $aRc$. However, $aRb$ but $\lnot(b R a)$, so the relation is not symmetric. 

        (Symmetric and transitive) Let $A$ be as above and 
        $$R_3 = \{(a,b),(b,a), (a,a), (b,b)\}$$
        Now every case of $xRy$ has a corresponding case $yRx$, and every case $xRy \land yRz$ has a corresponding $xRz$. However, $c$ is in no element in the relation, and thus $\lnot(cRc)$ which means the relation is not reflexive.

        \textbf{Comment:} The last combination I find the most interesting. If we were to have any cases of $xRy$ and $yRz$ where $x\neq z$, we would be required to have a relfexive relation. Only by completely omitting an element from every pair of the relation may we have a symmetric, transitive and non-reflexive relation. 

    \item
        (Reflexive) It is safe to say that any student has atleast one class with themselves, assuming no student has 0 classes. Under that assumption, $R$ is reflexive.

        (Symmetric) If student $a$ has a class with student $b$, then student $b$ has a class with student $a$. $R$ is symmetric.

        (Transitive) If student $a$ has a class, e.g., TMV027 with another student, $b$, and student $b$ has a class, e.g., DAT026, with another student, $c$, then there is no guarantee that $a$ and $c$ have any classes together. Say for example that both $a$ and $c$ have only on class. They may then both relate to $b$, who has at least to classes, but $a$ and $b$ can not be related.

        $\therefore R$ is not an equivalence relation.

    \item
        $$\mathcal{P}(S) = \{\emptyset, \{1\}, \{2\}, \{3\}, \{1,3\}, \{2,3\}, S\}$$
        \begin{align*}
            \subseteq =& \\
                       &\{(\emptyset, \emptyset), (\emptyset, \set{1}), (\emptyset, \set{2}), (\emptyset,\set{3}), \\
                       &(\emptyset, \set{1,2}), (\emptyset, \set{1,3}), (\emptyset, \set{2,3}), (\emptyset, S) \\
                       &(\set{1}, \set{1}), (\set{1}, \set{1,2}), (\set{1},\set{1,3}), (\set{1}, S) \\
                       &(\set{2}, \set{2}), (\set{2}, \set{1,2}), (\set{2},\set{2,3}), (\set{2}, S) \\
                       &(\set{3}, \set{3}), (\set{3}, \set{1,3}), (\set{3},\set{2,3}), (\set{3}, S) \\
                       &(\set{1,3}, \set{1,3}), (\set{1,3}, S), \\
                       &(\set{1,2}, \set{1,2}), (\set{1,2}, S), \\
                       &(\set{2,3}, \set{2,3}), (\set{2,3}, S), \\
                       &(S,S)\}
        \end{align*}
        %Fuck that set.

    \item
        \begin{enumerate}
            \item $R_1 \cap R_2 = \set{(a,b) \mid (a,b) \in R_1 \land (a,b) \in R_2}$. If both relations are reflexive, then $\forall x \in S . (x,x) \in R_1 \land (x,x) \in R_2$. Thus every such pair is also in the disjunction, and the disjunction is reflexive;
        \item If both relations are symmetric, then $\forall x,y \in S. (x,y) \in R_1 \Rightarrow (y,x) \in R_1$ and the same for $R_2$.
            Thus if a pair $(x,y)$ exists in both $R_1$ and $R_2$, then $(y,x)$ must also exist in both, and both $(x,y)$ and $(y,x)$ are in the disjunction, meaning ever pair in the disjunction has a reverse pair in the disjunction. So the disjunction is symmetric.
        \item By the same token as above, if any pairs $(x,y)$ and $(y,z)$ exists in both relations, then $(x,z)$ is also in both of the relations, and thus in the disjunctions.
        \item Both $R_1$ and $R_2$ must contain a pair $(x,x)$ for every $x \in S$, so their union must as well. The union is reflexive.
        \item If $(x,y)$ is in one of the relations, so is $(y,x)$. Thus if $(x,y)$ is in the union so is $(y,x)$, and the union is thus reflexive.
        \item If $(x,y)$ is in one of the relations but $(y,z)$ and $(x,z)$ is not, that does not violate transitivity of the relations. Identically, if $(y,z)$ is in the other relation, but $(x,y)$ and $(x,z)$ is not, that does not violate transitivity. However, that means that in the union both $(x,y)$ and $(y,z)$ exists, but not $(x,z)$, which violates transitivity. The union is thus not necessarily transitive.
        \end{enumerate}
        
    \item
        \begin{enumerate}
            \item Reflexive and transitive. Wouldn't matter if the empty set was included. This is true for the relation $\subseteq$ over any set.
            \item Reflexive and symmetric. Would change if the empty set was included, since that would break reflexivity because $\emptyset \cap \emptyset = \emptyset$. 
                
                Reflexivity comes from that for any nonempty set, it has at least one element in common with itself, and is thus not disjoint with itself. Symmetry comes from disjunction being commutative, so $A \cap B = B \cap A$ and thus if $A R B \Rightarrow A \cap B = B \cap A \neq \emptyset \Rightarrow B R A$.
            \item Symmetric and transitive. Wouldn't change if empty set was included. 

                The realtion is not refelxive since not every subset of $\mathbb{N}$ contains 1. It is symmetric since if $1 \in A \cap B$ then $1 \in B \cap A$ by commutativity. It is transitive since every set that contains 1 will relate to every other set that contains 1.
        \end{enumerate}

    \item
        \begin{enumerate}
            \item $\set{(0,3),(3,0),(1,2),(2,1)}$

                Symmetric
            \item $\set{(0,0), (0,1), (0,2), (0,3), (1,1), (1,2), (1,3), (2,2), (2,3), (3,3)}$

                Reflexive, transitive.

            \item $\set{(0,3),(3,0), (1,3),(3,1), (2,3), (3,2), (3,3)}$

                Symmetric

            \item $\set{(0,0), (0,2), (2,0), (1,1), (1,3),(3,1), (2,2), (3,3)}$

                Refelxive, symmetric, transitive (divides the set into the classes of odd and even numbers)

            \item $(S \times S) - \set{(2,3), (3,2), (3,3)}$

                $$\set{(0,0), (1,1), (2,2), (0,1), (1,0), (0,2),(2,0), (0,3), (3,0), (1,2), (2,1), (1,3), (3,1)}$$

                Symmetric
        \end{enumerate}

\end{enumerate}

\section{Functions}

\begin{enumerate}
    \item 
        \begin{enumerate}
            \item (Reflexive) $f(x) = f(x)$
                
                (Symmetric) $f(x) = f(y) \Rightarrow f(y) = f(x)$

                (Transitive) $f(x) = f(y) \land f(y) = f(z) \Rightarrow f(x) = f(z)$.
            \item
                Same as \ref{rel:square}.

            \item Assuming that (b) does not hold (since then $n=m=\infty$). (\textbf{Reader beware: I'm not completely sure on these two})

                \begin{enumerate}
                    \item Exactly $n$ classes: $\forall x,y \in A . x \neq y \Rightarrow f(x) \neq f(y)$, so every element in $A$ makes up its own class.
                    \item Anywhere between 1 and $n$ classes. It is possible that $\forall x,y \in A . f(x) = f(y)$ which means every element relates to every other, and its also possible that $\lnot\exists x,y . x \neq y \land f(x) = f(y)$, which means every element relates only to itself.
                \end{enumerate}
            
        \end{enumerate}

    \item
        A function $f : A \to B$ is

        surjective iff $\forall y \in B . \exists x \in A . f(x) = y$

        injective iff $\forall x,y \in A. f(x) = f(y) \Leftrightarrow x = y$

        \begin{enumerate}
            \item 
                The reverse function is such that $f((s)) = s$, meaning that the reverse function is its own inverse. Hence, the reverse function is an involution. Since there is an inverse of the function on the set, and it is closed under the reverse function, the function must be injective and surjective.
                
                For those in need of further proof, consider any string $x \in S$. We define $f(x) = x_{rev}$. Now, since the set is closed under the reverse function, $x_{rev} \in S$. Thus, $f(x)$ is defined for any choice of $x$. We can prove that $f$ is surjective by taking the assumption that $f$ is injective to be obvious (there is only one way to reverse a string, and for any reversed string $f(x)$ the value of $x$ can be easily deduced; a more rigorous proof of this should be failry easy, but I will not try to make one here). Since $f$ is injective, if we reverse any string in $S$ we get another element (which might be the same element, in case of palindromes) in $S$, which in turn can be reversed to render the original string. Thus every string in $S$ has exactly one string in $S$ which can be reversed to render the former, which makes $f$ surjective.
            \item
                $g$ is not injective. Proof by counterexample: $g(2,3) = g(3,2) = g(1,4)$. Each equality violates injectivity. There are in fact an infinite amount of elements in $\mathbb{R} \times \mathbb{R}$ that map to any element in $\mathbb{R}$ via $g$. $\forall a \in \mathbb{R} . g(x,y) = g(x-a, x+a)$.

                $g$ is surjective. For any $x \in R$, we can simply state that $g(\frac{x}{2}, \frac{x}{2}) = x$, so $g$ "covers" all of $\mathbb{R}$.
            \item
                Injective and not surjective. $\forall (m, n) \in \mathbb{N}  \times \mathbb{N} . (m \neq 0 \Rightarrow m = f(m-1)) \land (m - 1 = n - 1 \Leftrightarrow m = n)$. The idea here is that if you "shift" all the natural numbers one step to the right, you still end up with the exact same set of numbers, excpet for 0. Since no element in $\mathbb{N}$ maps to 0, it is not surjective

            \item
                Not injective, but surjective. Proof by counterexample for injectivity: $h(\text{hello}) = h(\text{horror})$. Proof for surjectiveness is any dictionary: all the letters in the alphabet have at least one word that begins with it.

            \item
                Neither. For any set with more than one element, there will be at least two elements in the power set which are sets that contain only one element, so it is not injective. And for any finite set $A$, the power set contains only subsets of $A$, the largest of which is $A$. So if $A$ cardinality $n$, then there is no subset of $A$ which has cardinality $n + 1$, but $n + 1 \in \mathbb{N}$, so it is not surjective.
        \end{enumerate}

    \item 
        $R_f = \set{(1,3), (2,1), (3,4), (4,2), (5,0)}$

        $f$ is injective, since the second element of every pair is unique compared to every other pair. It is not surjcetive, since no pair has 5 as its second element.

    \item
        \begin{enumerate}
            \item id, f, g
            \item id, g, k
        \end{enumerate}

    \item
        \begin{enumerate}
            \item $\set{1,2,3,4,5,74}$
            \item $\set{0, 0, 1, 2, 3, 72}$
            \item $f$ is injective since $\forall m,n \in \mathbb{N}. m + 1 = n + 1 \Leftrightarrow m = n$, which means that only one element in $\mathbb{N}$ maps to another. $f$ is not surjective since there $f(n) = 0 \Leftrightarrow n = -1 \not\in \mathbb{N}$.
            \item $g$ is not injective since $g(0) = g(1)$ and $0 \neq 1$. $g$ is surjective since two elements map to $0$, and $\forall n \in \mathbb{N} . n > 0 \Rightarrow g(n + 1) = n$, so all natural numbers have some other number mapping to them. 
            \item
                $g \circ f(n) = g(n+1) = \max\{0,n+1-1\} = \max\{0,n\}$. Since $\forall n \in \mathbb{N} . n \geq 0 $ and $\max\{a,b\}$ will return the element that is greater than or equal to the other, $g \circ f(n) = n$.

                Since as we stated above, $f$ is not surjective whereas $g$ is, we must be able to choose an element from the domain of $g$ that is not in the image of $f$, and then $f \circ g (n) \neq n$. There happens to be only one such element, 0. $f \circ g (0) = f(\max\{0,-1\}) = f(0) = 0 +1 = 1 \neq 0$.
        \end{enumerate}

    \item
        \begin{enumerate}
            \item 
                If $f(x) = 3x + 2$ then $\dfrac{f(x) - 2}{3} = x$. Thus $f^{-1}(y) = \dfrac{y - 2}{3}$. 
            \item No inverse exists, since $\forall x \in \mathbb{R}.f(x) = f(-x)$ and $x \neq -x$ in the general case, and thus $|\_|$ is not injective, and therefore it is not bijective, and therefore there can be no inverse. (We could also prove that $|\_|$ is not surjective by saying that $\forall x \in \mathbb{R} . |x| \geq 0$ and thus for any $y \in \mathbb{R}. y < 0$ there is a value $x$ such that $|x| = y$.

        \end{enumerate}


\end{enumerate}

\end{document}
