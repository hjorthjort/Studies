%General
\documentclass{article}
\usepackage[utf8]{inputenc}

%Symbols
\usepackage{commath}
\usepackage{amsmath}

%Formatting
\usepackage{hyperref}
\usepackage{amsthm}
\usepackage{alltt}
\newtheorem{definition}{Definition}[section]
\newtheorem{theorem}{Theorem}[section]
\newtheorem{example}{Example}[section]
\hypersetup{colorlinks=true}
\usepackage{graphicx}
\graphicspath{ {img/} }
\usepackage{caption}

\title{Intro}
\date{2016-03-21}
\author{Hjort}

\begin{document}
\maketitle
\pagenumbering{arabic}

\section{Automata}

\begin{definition}
    A mechanism that is relatively self operating, and acts according to a set of rules.
\end{definition}

Goes through a set of states, starting at an \textit{initial state}.

Example:

\begin{enumerate}
    \item State: OFF
    \item Push
    \item State: ON
    \item Push
    \item State: OFF
\end{enumerate}

Finite state machines have \textit{finite memory}.

We can prove correctness of an automata, and we will focus on these proofs a
lot in this course.

\begin{definition}
    There are deterministic and non-deterministic automata.
    In automata theory, a finite state machine is called a deterministic finite automaton (DFA), if
    each of its transitions is uniquely determined by its source state and input symbol, and
    reading an input symbol is required for each state transition.
    A nondeterministic finite automaton (NFA), or nondeterministic finite state machine, does not need to obey these restrictions. In particular, every DFA is also an NFA.
\end{definition}
Source: \href{https://en.wikipedia.org/wiki/Nondeterministic_finite_automaton}{Wikipedia}

\subsection{A concrete problem}

The classic: A man, a goat, a wolf and a cabbage should cross a river. The boat can hold only two of these.

We can model this as a finite automata.

We have a very specific problem with a set of rules, and a set of possible states. This lends itself well to modelling with an automata.

\section{Formal languages}

\begin{definition}
    In mathematics, computer science, and linguistics, a formal language is a set of strings of symbols that may be constrained by rules that are specific to it.
\end{definition}

\begin{example}{The alphabet}
    The alphabet of a formal language is the set of symbols, letters, or tokens from which the strings of the language may be formed; frequently it is required to be finite. The strings formed from this alphabet are called words, and the words that belong to a particular formal language are sometimes called well-formed words or well-formed formulas. A formal language is often defined by means of a formal grammar such as a regular grammar or context-free grammar, also called its formation rule.
\end{example}

Source: \href{https://en.wikipedia.org/wiki/Formal_language}{Wikipedia}

\begin{alltt}
    L -> A | B | ... | Z
    D -> 0 | 1 | 2 | ... | 9

    Nr -> D | D Nr
    Id -> L LLoD
    LLoD -> L LLoD | D LLoD | \( \in \)

    E -> E + E | E - E | E * E | E / E | (E) | Nr | Id
\end{alltt}

\subsection{Context-free grammar}

\begin{definition}
    A formal grammar is considered "context free" when its production rules can be applied regardless of the context of a nonterminal. No matter which symbols surround it, the single nonterminal on the left hand side can always be replaced by the right hand side. This is what distinguishes it from a context-sensitive grammar.
\end{definition}

Source: \href{https://en.wikipedia.org/wiki/Context-free_grammar}{Wikipedia}

\section{Parts of the course}

\subsection{Formal Proofs}

We often want to prove that our program/model/grammar is "correct" (meaning it satisfies a certain specification/property).

In particluar, you won't get a complex program/model/grammar right if you don't understand what is going on.

Often it is too expensive to prove correctness for everything we do, but it is often worthwile to prove that it certain core parts are absolutely correct

\subsubsection{Types of proofs}

\begin{enumerate}
    \item Deductive proof
    \item Proof by contradiction
    \item Proof by counterexample
    \item \textbf{Proof by (structural) induction} (big focus of course)
\end{enumerate}

\subsubsection{Regular Languages}

\textbf{Finite automate} were originally proposed in the 1940's as models of neural networks. Turned out to have many other uses!

Finite automate \textbf{define} regular languages.

In 1950's regular expressions were defined using mathematical notation.

\subsubsection{Context-free Languages}

We can only model things with finite memory with regular langagues, but we can add \textbf{more power} with CFL.

We do this by adding a stack that contains data and obtain a \textbf{push down automata}.

Context-free grammars were invented by Naom Chomsky(!) in mid 1950's.

These grammars play a core role in the description and design of programmings languages, interpreters and compilers.

\textbf{Both push-down automata and context-free grammars both define context-free languages.}

\subsubsection{Church-Turing Theorem}

Will be focus of only the last part of the course. We will spend about a week here.

\begin{enumerate}
    \item Recursive functionos by Stephen Kleene (after ideas of Kurt Gödel).
    \item $\lambda$-calculus
\end{enumerate}

The halting problem

Tractable (polynomial) vs intractable (NP-Complete / NP-Hard) problems

\section{Learning Outcomes}

\begin{enumerate}
    \item Explain and manipulate the concpets of automata theroy and formal languages.
    \item understand the equivalnce between (non-)deterministic finite automate and regular expressions
    \item good understanding of the power and limitations of regular languages and CFL.
    \item Prove properties of languages, grammars and automata with formal mathematical methods.
    \item design automata, regexes and CFG:s \textbf{acceptin} or \textbf{generating} a certain language
    \item describe the language accepted by an automata or genereted by a regex or CFG
    \item  Simplify automata and CFG:s
    \item Determine if a certain word belongs to a language
    \item Define Turing Machines peforming simple tasks.
    \item Differentiate and manipulate formal descriptions of langagues, automata and grammars
\end{enumerate}

\end{document}
