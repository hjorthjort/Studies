\documentclass{article}

\usepackage{amsmath}
\usepackage{amssymb}
\usepackage{amsthm}
\newcommand{\Mod}[1]{\ (\mathrm{mod}\ #1)}
\newcommand{\z}{Z^*_p}
\newcommand{\ord}{\text{ord}}
\newcommand{\qr}{{quadratic residue}}
\newcommand{\qnr}{{quadratic non-residue}}
\newcommand{\Qr}{{Quadratic residue}}
\newcommand{\Qnr}{{Quadratic non-residue}}

\newtheorem{theorem}{Theorem}[section]
\newtheorem{algorithm}[theorem]{Algorithm}
\newtheorem{lemma}[theorem]{Lemma}

\title{\Qr}
\author{Rikard Hjort\\\texttt{hjortr@student.chalmers.se}}
\date{\today}


\begin{document}
\maketitle
\section{Algorithm for finding non-residue}

If $p = 2$, then $\z = \{1\}$, and since 1 is a a \qr, $\z$ contains no \qnr
s. We will therefore restrict ourselves to the case when $p \geq 3$.

\begin{lemma}\label{half}
  Exactly half the elements in $\z$ are \qr s.
\end{lemma}
\begin{proof}
  Since $\z$ is cyclic, it has at least one generator, $g$. By proposition 2,
  $g^k$ is a \qr~iff $2 \mid k$. Since $\ord(g) = |\z|= p-1$, and $p$ is odd,
  there are precisely $\frac{p-1}{2}$ values of $k$ for which $g^k$ is a \qr,
  and as many that are \qnr s.
\end{proof}

This means that random sampling is efficient, which makes it simple to devise an algorithm.
\begin{algorithm}
  \hfill\break
  \begin{description}
  \item[Input] $p$, a prime number $\geq 3$.
  \item[Output] $a$, a \qnr.
  \end{description}
  \begin{enumerate}
  \item\label{start} Pick a random integer $1 < a < p$. (Skip 1 as it is always
    a \qr.)
  \item Calculate $a^{\frac{p-1}{2}} = r$.
    \begin{enumerate}
    \item If $r = 1$, repeat from step \ref{start}.
    \item Else, return $a$.
    \end{enumerate}
  \end{enumerate}
\end{algorithm}

The algorithm will only return correct values (it is a Las Vegas algorithm),
by proposition 4. The algorithm will probabilistically terminate, since
there is $\frac{p-1}{2}$ \qnr s in $\z$.

\paragraph{Runtime analysis}

Exponentiation can be done in polynomial time in $\log p$ with binary
exponentiation, and decrementing $p$ and halving it can be done in linear time
or better, since it is simply a matter of flipping the least significant bit and
shifting. Testing eqaulity with 1 can be done in constant time.

The algorithm runs an expected number of 2 times, since by lemma \ref{half} it
has a $\frac{1}{2}$ probability to find a \qnr at every iteration. The total
runtime is donimated by exponentiation, and thus runs in polynomial time in
$\log p$.

\section{Find root of $a \in \z$ given $a$ and a \qnr~$b$}

We start with some helpful findings. 

\begin{lemma}\label{2roots}
  Every \qr~of $\z$ has exactly 2 roots, $b$ and $c$, with $b = -c$.
\end{lemma}

\begin{proof}
  If $a \in \z$ is a \qr, and $a = g^{2k}$ for some generator $g$, then also $a
  = (p-g^k)^2 = p^2 - 2pg^k + g^{2k} = g^{2k}$. Also, $g^k$ and $p - g^k$ are
  different, since $g^k = p - g^k \Leftrightarrow p = 2g^k \Rightarrow 2 \mid p$
  which is not possible, since $p$ is odd. Furthermore,
  $$b^2 = c^2 \Leftrightarrow c^{-1}bc^{-1}b = 1$$
  which means $c^{-1}b$ is its own inverse (keep in mind $\z$ is Abelian).
  Recall that that $|O_2| = \phi(2) = 1$, and $\ord(-1) = 2$, and thus $O_2 =
  \{-1\}$. Thus, only one element is its own inverse, namely $-1$. This means
  that $c^{-1}b = -1$, which in turn means $b=-c$. Thus, $a$ can have
  precicesly 2 roots, whith one being $b$, and the other $-c = p - c$.
\end{proof}

\begin{lemma}\label{plusminus}
  For $a \in \z$, $a^{\frac{p-1}{2}}= \begin{cases}1 \text{ if } a \text{ is a
      \qr } \\ -1 \text { if } a \text{ is a \qnr}\end{cases}$
\end{lemma}
\begin{proof}
  If $a$ is a \qr, then for a generator $g$ such that $g^{2k}=a$ we can express
  $a^{\frac{p-1}{2}} = g^{2k\frac{p-1}{2}} = g^{k(p-1)} = 1$. If $a$ is a \qnr,
  then for a generator $g$ such that $g^{2k + 1} = a$, we can express
  $a^{\frac{p-1}{2}} = g^{(2k+1)\frac{p-1}{2}} = g^{k(p-1)}g^{\frac{p-1}{2}}=
  g^{\frac{p-1}{2}} \not = 1$ since $\ord(g) \geq \frac{p-1}{2}$. On the other
  hand, $g^{\frac{p-1}{2}}$ is a root of $1$, because squaring it gives $1$.
  Thus, $g^{\frac{p-1}{2}} = -1$ by lemma \ref{2roots}.
\end{proof}

We divide the solution into three cases.

\subsection{$p \equiv 3 \mod 4$}
For the case $p \equiv 3 \mod 4$ we need not bother with $b$. It suffices to
note that $a^{\frac{p-1}{2}} \equiv 1 \mod p \Rightarrow a^{\frac{p-1}{2}}a
= a^{\frac{p+1}{2}} \equiv a \mod p$, and since $p + 1 \equiv 0 \mod 4$, we can
safely divide $p+1$ by 4, and get $a^{\frac{p+1}{4}}$ as a root of $a$, and
$p - a^{\frac{p+1}{4}}$ as the other.

\subsection{$p \equiv 5 \mod 8$}

If $p \equiv 1 \mod 4$, there are two more cases: $p \equiv 1 \mod 8$ and $p
\equiv 5 \mod 8$. The second case is simpler. $a^{\frac{p-1}{2}} = 1$ means
$a^{\frac{p-1}{4}}$ is a root of 1 (we can safely divide by 4, as $p -1 \equiv 4
\mod 8$), and thus $\pm 1$. If $a^{\frac{p-1}{4}} = 1$, we are done by the same
trick as before: $a^{\frac{p-1}{4}}a = a^{\frac{p+3}{4}} = a$, so we return
$a^{\frac{p+3}{8}}$, which is a root of $a$ ($p+3 \equiv 0 \mod 8$, so again,
the division in the exponent is safe).

However, if $a^{\frac{p-1}{4}} = -1$, we
can make use of $b$. $b^{\frac{p-1}{2}} = -1$ by lemma \ref{plusminus}, so

$$a^{\frac{p-1}{4}} * b^{\frac{p-1}{2}} = 1 \Rightarrow a^{\frac{p+3}{4}} *
b^{\frac{p-1}{2}} = a $$

Dividing by 2 again gives us that $\pm a^{\frac{p+3}{8}} * b^{\frac{p-1}{4}}$ is
the roots of $a$.


\subsection{$p \equiv 1 \mod 5$}
I have not yet solved this case.


\end{document}
