\documentclass{article}

\usepackage{amsmath}
\usepackage{amssymb}
\newcommand{\Mod}[1]{\ (\mathrm{mod}\ #1)}
\newcommand{\z}[1]{Z^*_{#1}}
\newcommand{\ord}{\text{ord}}

\title{Generators of $Z^*_p$}
\author{Rikard Hjort}
\date{\today}

\begin{document}

\maketitle

\section{Generators of $Z^*_{17}$}

Since $\phi(16) = 8}$ we expect to find 8 generators. The table below shows
which they are. The top column lists the generators, and each row represents
taking the generator to the power $2, 3, 4, ... 16$.

\vspace{1em}
\begin{tabular}{l | l l l l l l l l}
a      &  3 &  5 &  6 &  7 & 10 & 11 & 12 & 14 \\ \hline
a^{2 } &  9 &  8 &  2 & 15 & 15 &  2 &  8 &  9 \\
a^{3 } & 10 &  6 & 12 &  3 & 14 &  5 & 11 &  7 \\
a^{4 } & 13 & 13 &  4 &  4 &  4 &  4 & 13 & 13 \\
a^{5 } &  5 & 14 &  7 & 11 &  6 & 10 &  3 & 12 \\
a^{6 } & 15 &  2 &  8 &  9 &  9 &  8 &  2 & 15 \\
a^{7 } & 11 & 10 & 14 & 12 &  5 &  3 &  7 &  6 \\
a^{8 } & 16 & 16 & 16 & 16 & 16 & 16 & 16 & 16 \\
a^{9 } & 14 & 12 & 11 & 10 &  7 &  6 &  5 &  3 \\
a^{10} &  8 &  9 & 15 &  2 &  2 & 15 &  9 &  8 \\
a^{11} &  7 & 11 &  5 & 14 &  3 & 12 &  6 & 10 \\
a^{12} &  4 &  4 & 13 & 13 & 13 & 13 &  4 &  4 \\
a^{13} & 12 &  3 & 10 &  6 & 11 &  7 & 14 &  5 \\
a^{14} &  2 & 15 &  9 &  8 &  8 &  9 & 15 &  2 \\
a^{15} &  6 &  7 &  3 &  5 & 12 & 14 & 10 & 11 \\
a^{16} &  1 &  1 &  1 &  1 &  1 &  1 &  1 &  1 \\
\end{tabular}

\section{Prove propositions}
\subsection{If $k \mid p-1$ then $|O_k| = \phi(k)$} 
\paragraph{Proof:} Any multiplication group modulo $n$ is cyclic if $n$ is a
prime (generally, if it is prime to any positive power). Since $\z p$ is thus
cyclic, it has at least one generating element. Let $a$ be any generator of $\z
p$, and $1 \leq d \leq p-1$. $\gcd(d, p-1) \mid d$ and $\gcd(d, p-1) \mid p-1$
which means that if we set $k=(p-1)/\gcd(d,p-1)$ then $dk = m(p-1)$ for $m = d /
\gcd(d, p-1)$. By Fermat's little theorem, $a^{dk} = a^{m(p-1)} = 1$. Since
$(a^d)^k = 1$, the order of $a^d$ is some divisor of $k$. But by our choice of
$k$ it contains only the prime factors not common to $d$ and $p-1$. Dividing by
any of these prime factors (the only valid division of $k$), obtaining $k'$
would result in a number for which $p -1 \nmid dk'$, since there is some prime
factor in $p-1$ missing in $dk'$. Thus, the smallest possible number we can
multiply $d$ with to obtain a multiple of $p-1$ is $k$, which means the order of
$a^d$ is exactly $k$. Now, we will examine how many possible choices there are
of $d$ for which $k$ remains unchanged. Set $d = fk, 1 \leq f \leq (p-1)/k =
\gcd(d, p-1)$. The choices of $f$ for which $k$ remains unchanged are exactly
those which do not contain any of the prime factors of $k$, or $\gcd(f, k) = 1$.
The number of such choices are exactly $\phi(k)$ by definition. Thus, we have
proven that the number of elements $a^d$ of order $k$ is $\phi(k)$, $Q.E.D.$.

\subsection{$\forall n > 1, \frac{\phi(n)}{n} = \Omega\left( \frac{1}{\log n} \right)$}

We begin by observing that
$$\phi(n) = \prod_{(p, k) \in P(n)}p^k\left(1-\frac{1}{p}\right) = n \prod_{(p, k) \in P(n)}1-\frac{1}{p}$$

\noindent where $P(n) = \{(p,k) \mid p^k\text{ is in the prime factorization of } n\}$. This
in turn means that

$$ \frac{\phi(n)}{n} = \prod_{(p,\_) \in P(n)}1 - \frac{1}{p}$$

Let $p_i$ be any prime dividing $n$. We know that $p_i \geq 2$, and thus $1
-\frac{1}{p} \geq \frac{1}{2}$. This also means that $|P(n)|$ can not be larger
than $\log_2 n$, which replacing each $p_i$ with 2 in
$$ p_1p_2...p_{|P(n)|} \leq n = 2^{\log_2 n}$$
\noindent should make evident. Thus, 

$$ \frac{\phi(n)}{n} = \prod_{(p,\_) \in P(n)}1 - \frac{1}{p} \quad \geq
\prod_{i=1}^{\log_2 n} \frac{i}{i+1} = \frac{1}{2}*\frac{2}{3} \dots * \frac{\log
  _2 n}{1 + \log_2 n } = \frac{1}{1 + log_2 n} $$

\noindent where the final equality comes from pairing numerators and
denominators in the expansion of the product, with only the first numerator and
final denominator surviving.

We multiply by a constant, e.g., $\frac{1}{2}$, in the following inequality, and
see that $$\frac{1}{1 + \log_2 n} \geq \frac{1}{\log_2 n}* \frac{1}{2} $$

\noindent since $\log n \geq 1$ for all $n \geq 2$. Thus, $\frac{\phi(n)}{n} =
\Omega\left( \frac{1}{\log n} \right), Q.E.D.$


\subsection{Given the prime factorization of $p-1$, it can be tested whether an
  $x \in \z p$ is a generator in polynomial time in $\log p$}

If $x \in \z p$ is a generator, it must have order $p -1$. Assume $x$ is
\textbf{not} a generator. By Proposition 1 (and Lagrange's theorem), $\ord(x)
\mid p - 1$, (but $p-1 \nmid \ord(x)$). Thus $\ord(x)$ shares some, but not all,
prime factors with $p-1$. Assume $p'$ is one prime factor of $p -1$ but not of
$\ord(x)$. Then $\ord(x) \mid \frac{p-1}{p'}$, so $x^{(p-1)/p'} = 1$. To find out
if $x$ is a generator, we must therefore test this with all distinct primes in
the prime factorization of $p-1$. If $\ord(x) = p-1$ on the other hand, then
$x^{(p-1)/k} \not = 1$ for any positive $k$, by the definition of order. So if
we perform the above test with all distinct primes in the factorization of $p-1$
and find none of them gives $x^{(p-1)/p'} = 1$, then $\ord(x) = p-1$, and $x$ is
a generator.

Dividing $p-1$ by $p'$ can be done in polynomial time in $\log p$. To perform a
single test, i.e., exponentiating $x$ in $\z p$, can be done in polynomial time
in $\log p$ by binary exponentiation. Finally, the number of prime factors of
$p-1$ can be at most $\log p$, since $p-1= 2^{\log (p-1)}=p_1p_2...p_\ell$,
where the last expression is the prime factorization, with each $p_i \geq 2$.

Thus, testing whether $x$ is a generator is done by performing a single test by
dividing $p-1$ by a single prime factor and exponentiating in $O(P(\log p))$
time where $P(x)$ is a polynomial, and the test must be repeated up to $O(\log
p)$ times, yielding a $O(\log p) * O(P(\log p))$ time bound, which is obviously
polynomial in $\log p,$ $Q.E.D.$.

% Look from page 413 and 414 in book.

\section{An algorithm for finding generators of $\z p$}
% Verify with book.

\begin{description}
\item[Input] $p$, a prime number and $F$, the facorization of $p-1$. Let $n =
  log p$ be the length of $p$, and thus the input size.
  \item[Output] $g$, a generator of $\z p$
\end{description}

\paragraph{Algorithm:}
\begin{enumerate}
  \item\label{random} Randomly select an number $g$ such that $1 \leq g \leq p - 1$.
   \item Test, using the approach in the proof of Proposition 3, whether $g$ is
     a generator of $\z p$.
   \item If $g$ is a generator, return $g$. Else, repeat from step \ref{random}.
\end{enumerate}

The algorithm will only return correct values, since the final step only returns
when we find exactly such an element that we are looking for, and not otherwise.
The algorithm will probabilistically terminate, since there are always a
generator of $\z p$ in $\z p$.

\paragraph{Runtime analysis}

Randomly selecting an element can be done in constant time. Testing if it is a
generator can be done in $\log p$ time by Proposition 3. There are $\phi(p-1)$
generators by Proposition 1, and $frac{\phi(p-1)}{p-1} \geq \frac{1}{\log p-1}$
by Proposition 2, which means the so the chance of finding a generator at any given
try is larger than $\frac{1}{\log p-1}$, because we are looking for $\phi(p-1)$
elements among $p-1$ elements. Thus, we expect to repeat a loop which takes
$O(\log p)$ time for $\log p - 1 < \log p$ repetitions, so the algorithm's
runtime $T = O(\log p) * O(\log p)$, which is $O(n^2)$.

  


\end{document}

% \subsection{$\forall n > 1: \frac{\phi(n)}{n} \geq \frac{1}{\log n}$}
% 
% \paragraph{Proof:} 
% $$\frac{\phi(n)}{n} = \prod_{p \mid n}\left ( 1 - \frac{1}{p} \right )$$
% 
% \noindent for $p$ each prime factor in $n$. 

% Alternate explanation:
% Assume $\z p$ is cyclic (TODO: Prove it). Then there is at
% least one element $a \in \z p$ such that ${a, a^2, ..., a^{p-1}} = \z p$, by
% definition of a cyclic group. Remember that for any \textit{additive} group
% $Z_n$, every element $b$ such that $gcd(b,n) = 1$ is a generator of the group,
% and only those elements. Thus, for each $d$, $a^d$ is a generator of $\z p$ if
% $gcd(d, p-1) = 1$ simply because then $p - 1 \mid md$ iff $p-1 \mid m$, and thus
% $a^{md} = 1$ iff $m$ is a multiple of $p-1$, because by Fermat's theorem,
% $a^{md} = 1$ iff $p-1 \mid md$. This in turn means that for any $m < p-1$,
% $a^{md} \not= 1$. The number of such $d$, for which the order is $p-1$, is
% exactly $\phi(p-1)$ by definition.
% 
% On the other hand, for any $d$ such that $\gcd(d, p-1) > 1$, let $g = \gcd(d,
% p-1)$. By definition, $g \mid p - 1$ and $g \mid | d$, and thus $(a^d)^{(p-1)/g}
% = 1$. The order of $a^{d/g}$ is $(p-1)/g$ by the same reasoning as above. The
% number of such $d$ is $\phi{(p-1)/g}$, because for any $x,y$,
% $\gcd(x/\gcd(x,y),y/\gcd(x,y)) = 1$.
% 

