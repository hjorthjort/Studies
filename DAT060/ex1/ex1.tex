\documentclass{article}
\usepackage[shortlabels]{enumerate}
\usepackage{fitch}
\usepackage{amsmath}

\author{Rikard Hjort}

\begin{document}
\maketitle
\section*{Problem 1}
\begin{enumerate}[(1)]
    \item Depending on how we interpret ``if six turns out to be nine'', and whether we care to consider Jimmy's and Johnny's drinking separately.

        In the simplest case, we can define:

        \begin{align*}
            p &:= \text{six turns out to be nine} \\
            q &:= \text{Jimmy and Johnny drink wine} \\
        \end{align*}
        Then the proposition is $$p \rightarrow q$$.

        Another reading I see as possible is that $p$ means ``six $ \leftrightarrow$ nine'' so another possibility is, given
        \begin{align*}
            p_6 &:= \text{six} \\
            p_9 &:= \text{nine} \\
        \end{align*}
        giving 
        $$(p_6 \rightarrow p_9) \land (p_9 \rightarrow p_6) \rightarrow q$$

        Lastly, we could consider Jimmy and John separately.
        \begin{align*}
            q_{Jimmy} &:= \text{Jimmy drinks wine} \\
            q_{Johnny} &:= \text{Johnny drinks wine} \\
        \end{align*}
        giving the fullest version I can think of for the formula, namely
        $$(p_6 \rightarrow p_9) \land (p_9 \rightarrow p_6) \rightarrow q_{Jimmy} \land q_{Johnny}$$

    \item
        One way to do this is to note that any one of the logicians liking pets ensures none of the other could like pets. Thus, we can define
        \begin{align*}
            p &:= \text{Miranda likes pets} \\
            q &:= \text{Haskell likes pets} \\
            r &:= \text{Kleen likes pets} \\
        \end{align*}
        and, remembering that de Morgan's laws gives us $ \lnot(\phi \lor \psi) \leftrightarrow \lnot \phi \land \lnot \psi)$ and make the formula

        $$(p \rightarrow \lnot(q \lor r)) \land (q \rightarrow \lnot(p \lor r)) \land (r \rightarrow \lnot(p \lor q))$$

\end{enumerate}

\newpage

% Fuck this
% \section*{Problem 2}
% 
% For simple typesetting, I use Fitch style notation, which I find is very similar to the notation system in the course book.
% 
% \begin{enumerate}[(1)]
%     \item
% 
% \begin{equation*}
%     \begin{fitch}
%      p \lor (q \lor r) & premise \\
%  \fa p & assumption \\
%  \fa p \lor q & \lor i_1 2 \\
%  \fa (p \lor q) \lor r & \lor i_2 3 \\
%  \fa (q \lor r) & assumption \\
%  \fa \fa q & assumption \\
%  \fa \fa \fa p \lor q & \lor i_2 6\\
%  H H H & Main Proof Step
%     \end{fitch}
% \end{equation*}
% \end{enumerate}
% 
% 
\end{document}
