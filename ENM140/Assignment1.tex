\documentclass{article}
\usepackage{fullpage}
\usepackage{url}

\title{Self-assesed property tax}
\author{Rikard Hjort\\
  hjortr\\
hjortr@student.chalmers.se}

\begin{document}
\maketitle

\paragraph{Overview}
Consider a property tax scheme, where a land owner specify land value themselves,
and are taxed 3~\% of the value every year.\footnote{Sometimes called Harberger tax. Inspired by \url{https://vitalik.ca/general/2018/04/20/radical_markets.html}} The value is reported to the government, and is kept constant until a new value is reported.

The posted value of land is then effectively an asking price. Anyone can buy the land (and the structures on it)
for the asking price. The transfer of
ownership happens immediately, and the new owner may set any price they choose. The current land users then then have 3 months to vacate.

\paragraph{Players}
The game is over 1 piece of land, with 2 players: a current owner,
and a market, consisting of all other prospective land owners.\footnote{This captures the rational
game theory actors better than more granular approaches, since we may well assume the
market is, in some sense, rational, and will offer only a price which the market
(meaning some entity) thinks it can get back from development of the property.}
Call these players ``land-lord'' and ``market''. We assume the property starts
out in the hands of the land-lord. 

\paragraph{Actions and strategies}
Model this as a sequential game. In every round, the land-lord sets an asking
price, and the market places a bid. The player with the highest bid takes
control of the property. If the market wins, the land-lord gets the full value
of their bid.

One strategy for the land-lord may be to set a high price (higher than the
expected utility), hoping someone else will pay that price, giving a profit.
Another strategy is to set a very low price, reducing tax costs, hoping no one
will buy, but riskiing losing the property, and having to buy it back later.

\paragraph{Utilities}
Every month they retain ownership, the landlord get 30~000~SEK of utility they, and pay
0.25~\% of their current asking price in tax.

Losing the property, and having it be lost for more than 3 rounds (``getting
vacated'') costs 20~000~SEK. If the land-lord buys back within 3 rounds, no
switching cost is incurred.

Further, the utility of owning of the property increases for the owner
the longer they hold it\footnote{This could be senitimental value, or value of
  continued development towards long term plans}, so every 12 rounds of
ownership the utility increases by 3~\%. If the land-lord is vacated, the
utility goes back to 30~000~SEK per round.

The market expects to be able to get 35~000~SEK of utility per month, with the
same switching cost and projected value increase.

The game is played for 240 rounds (20 years), to avoid infinite growth problem.
\end{document}

ideas
\begin{itemize}
\item basic auction: first payer, second payer, third payer. Ye auctions: the proof for optimal bidding in first price is pretty elegant, and then you can extend that to second price, third price
\item Bayesian games
\item Hawk-Dove (evolutionary games, where players can reproduce based on their success or they die / survive)
\item Signalling: Education signalling \& insurance signalling (may be called
  ``screening game'')
  \end{itemize}
    
Goal: find a game, predict outcome, see if it matches observations from real
life, update parameters.
