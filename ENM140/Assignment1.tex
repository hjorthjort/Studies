\documentclass{article}
\usepackage{fullpage}
\usepackage{url}

\title{Self-assesed property tax}
\author{Rikard Hjort\\
  hjortr\\
hjortr@student.chalmers.se}

\begin{document}
\maketitle

\paragraph{Overview}
There are a few different ways to have to tax property, for example the Georgian tax, which
is based purely on land value (thus not penalizing development), and tax based
on full property value. Both of these rely on a government assessing property value.

Consider a different scheme, where a land owner specify land value themselves,
and are taxed 3~\% of the value every year.
  \footnote{Sometimes called Harberger tax. Inspired by \url{https://vitalik.ca/general/2018/04/20/radical_markets.html}}
 The value is reported to the government, and is kept constant until a new value is reported.

There is a simple incentive not to cheat: the posted value of land is
effectively an asking price. Anyone can buy the land (and the structures on it)
for the asking price, whihc would force the previous owner out. The transfer of
ownership happens immediately, and the new owner may set any price they choose.
The current land users then then have 3 months to vacate.

\paragraph{Players}
We consider the game being for a single piece of land, and we can then imagine
exactly 2 players: a current owner (paying tax), and a market, consisting of all
other entities, including the government. This also better captures the rational
game theory actors than more granular approaches, since we may well assume the
market is, in some sense, rational, and will offer only a price which the market
(meaning some entity) thinks it can get back from development of the property.

\paragraph{Actions and strategies}
We can look at this as a sequential game. In the first round, the owner sets an
asking price. Then every round (we can imagine them representing a month), and
the market may choose to buy the property at the old asking price, and set a new
asking price. If they don't the owner may change the price (up or down).

One strategy may be to set a high price (higher than the expected utility for
the owner), hoping someone else will pay that price, giving a profit. Another
strategy is to set a very low price, reducing tax costs, hoping no one will buy,
but riskining losing the property, and having to buy it back.

I can't really see any mixed strategies that make sense.

\paragraph{Utilities}
Say that an owner expects his land yields him a montly (net) income or utility of
30~000~SEK. Losing the property and being vacated for some time means losing this income, but also
some ``switching costs'' of administration, ruined planning, etc., which we
value at 10~000~SEK. This cost is incurred 3 iterations (months) after losing
ownership, unless they buy back. Further, the value of the property increases for the owner
the longer they hold it\footnote{This could be senitimental value, or value of
  continued development towards long term plans}, so every 10 months of future ownership increases the
value of the property (to the owner) by 10~\%.

The market has similar valuations for the property, but let's say there is one
player who value the property slightly higher, who thinks they can get
35~000~SEK of utility per month, with the same switching cost and projected
value increase.


\end{document}

ideas
\begin{itemize}
\item basic auction: first payer, second payer, third payer. Ye auctions: the proof for optimal bidding in first price is pretty elegant, and then you can extend that to second price, third price
\item Bayesian games
\item Hawk-Dove (evolutionary games, where players can reproduce based on their success or they die / survive)
\item Signalling: Education signalling \& insurance signalling (may be called
  ``screening game'')
  \end{itemize}
    
Goal: find a game, predict outcome, see if it matches observations from real
life, update parameters.
